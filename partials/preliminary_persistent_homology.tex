\section{Preliminary}

\subsection{Persistent homology}

One of the central goals in data analysis is to describe and classify the structure and properties of given data. This objective closely parallels a major aim in topology, that is, the classification and characterization of topological spaces. Consequently, techniques developed in these in topology are expected to be applicable to data analysis as well. Topological data analysis (TDA) is a framework motivated by this perspective, and persistent homology \cite{edelsbrunner2002topological} has emerged as one of its most prominent methods.

Persistent homology studies the ``shape'' of data by converting the data into a parametric family of spaces and analyzing how the topology of these spaces changes along the parameter. The relevant topological features are quantified via homology groups.

The aim of this work is to investigate Turing patterns generated by reaction-diffusion systems through the lens of persistent homology. To this end, in this section we briefly review the notion of homology groups, introduce the definition of persistent homology for real-valued functions on topological spaces, and present the definition of persistence diagrams, which serve as a concise summary of persistent homology. We also discuss stability results connecting the input functions to their associated persistence diagrams.

For more comprehensive treatments of persistent homology, we refer the reader to \cite{dey2022computational}.

\subsubsection{Informal description of homology group}

The persistent homology employed in this study is defined on the basis of homology groups, which capture the ``shape'' of a topological space. In this subsection, we do not delve into the rigorous construction of homology groups; instead, we provide intuitive descriptions and simple examples sufficient for understanding persistent homology. For precise definitions and the general theory, we refer the reader to standard texts in geometry and topological data analysis. For instance, \cite{dey2022computational} offers a detailed exposition of homology groups in the context of persistent homology.

Homology groups extract geometric features of a topological space and encode them in algebraic form. More concretely, for a topological space $X$, the $p$-th homology group $H_p \left( X \right)$ is an Abelian group whose elements represent algebraic counterparts of geometric features present in $X$. Such features include connected components, holes, and voids within the space $X$. These geometric features correspond to elements of the homology group called homology classes.

Several constructions of homology groups are known; in this work, we adopt singular homology with coefficients in $\mathbb{Z}_2 = \mathbb{Z} / 2 \mathbb{Z}$, which is commonly used for general topological spaces.

Finally, we recall the following fundamental fact concerning pairs of topological spaces related by inclusion.
\begin{prop}\label{prop:natural-linear-map-of-inclusion}
  Let $X$ and $Y$ be topological spaces with $X \subset Y$. Then, for any integer $p \geq 0$, there exists a natural linear map $\iota_*: H_p \left( X \right) \to H_p \left( Y \right)$ between the homology groups of $X$ and $Y$.
\end{prop}

The natural linear map given in this proposition describes how geometric features represented in the homology of $X$ are inherited by the homology of $Y$. This map plays a central role in the definition of persistent homology.

\subsubsection{Definition of persistent homology and persistence diagram}

Next, we define persistent homology for real-valued functions on topological spaces. As noted earlier, persistent homology analyzes the ``shape'' of a one-parameter family of spaces by tracking how their homology changes along the parameter. For a function on a topological space, such a family is naturally generated by its sublevel set filtration. We begin by recalling the definition of this filtration.

We first introduce the notation used below. Let $p \geq 0$ be an integer, let $\bar{\mathbb{R}} = \mathbb{R} \cup \Set{ \infty }$, and let $X$ be a topological space equipped with a real-valued function $f: X \to \mathbb{R}$. For a real number $a \in \mathbb{R}$, the sublevel set of $f$ at threshold $a$ is defined by $\Set{ f \leq a } = \Set{ x \in X | f \left( x \right) \leq a }$.

\begin{dfn}
  Let $X$ be a topological space and $f: X \to \mathbb{R}$ a real-valued function. The collection of sublevel sets $\mathcal{F} \left[ f \right] = \left( \Set{ f \leq a } \right)_{a \in \mathbb{R}}$ is called the sublevel set filtration of $f$.
\end{dfn}

\begin{rem}
  A simple concrete example of a sublevel set filtration can be constructed as follows. Let $\left( X, d \right)$ be a metric space, and let $C$ be a discrete subset of $X$. Define a function $d_C: X \to \mathbb{R}$ by, for any $x \in X$,
  \begin{align*}
    d_C \left( x \right) = \inf_{c \in C} d \left( x, c \right)
  \end{align*}
  Then, for any $a \in \mathbb{R}$,
  \begin{align*}
    \Set{ d_C \leq a }
    &= \Set{ x \in X \mid d_C \left( x \right) \leq a }\\
    &= \bigcup_{c \in C} \Set{ x \in X \mid d \left( x, c \right) \leq a }.
  \end{align*}
  Therefore, the sublevel set filtration $\mathcal{F} \left[ f_C \right]$ is, as in Figure \ref{fig:example-sublevel-set-filtration}, a sequence of spaces obtained by thickening each point of $C$.
\end{rem}

\begin{figure}[h]
  \centering
  \begin{tikzpicture}[scale=0.5]
    \def\R{2.0}
    \def\r{0.6}
    \def\nptsB{100}
    \def\nptsT{40}
    \def\dx{6.8}
    \def\pointsize{0.05}
    \def\labelsep{1.8}

    \pgfmathsetmacro{\rlabel}{- \R - \labelsep}
    \pgfmathsetmacro{\llabel}{- \R - \labelsep - 1}

    \pgfmathsetseed{20260105}

    \foreach \i in {1,...,\nptsB}{
      \pgfmathsetmacro{\ang}{360*rnd}
      \global\expandafter\edef\csname angleb\i\endcsname{\ang}
    }

    \foreach \i in {1,...,\nptsT}{
      \pgfmathsetmacro{\ang}{360*rnd}
      \global\expandafter\edef\csname anglet\i\endcsname{\ang}
    }

    \newcommand{\drawEight}[3]{%
      \coordinate (Cb) at (#1, 0);
      \coordinate (Ct) at (#1, {\R + \r + 0.5});

      \foreach \i in {1,...,\nptsB}{
        \path (Cb) ++(\csname angleb\i\endcsname:\R) coordinate (P);
        \fill[black] (P) circle[radius=#2];
      }

      \foreach \i in {1,...,\nptsT}{
        \path (Ct) ++(\csname anglet\i\endcsname:\r) coordinate (Q);
        \fill[black] (Q) circle[radius=#2];
      }

      \node at (#1, \rlabel) {#3};
    }


    \drawEight{0}{\pointsize}{$a = 1$}
    \drawEight{\dx}{4.4 * \pointsize}{$a = 5$}
    \drawEight{2 * \dx}{7.8 * \pointsize}{$a = 8$}
    \drawEight{3 * \dx}{12 * \pointsize}{$a = 12$}

    \node at (-6, 0) {$\Set{ f_C \leq a }$};

  \end{tikzpicture}

  \caption{A diagram of the sublevel set filtration of $d_C$ for a discrete point set $C$ uniformly distributed on an eight-like figure in the plane.}
  \label{fig:example-sublevel-set-filtration}
\end{figure}

For any real numbers $a \leq b$, we have $\Set{ f \leq a } \subset \Set{ f \leq b }$. Hence, by Proposition \ref{prop:natural-linear-map-of-inclusion}, for each integer $p \geq 0$ and $a \leq b$, there is a natural linear map $\left( \iota_{a, b} \right)_*: H_p \left( \Set{f \leq a} \right) \to H_p \left( \Set{f \leq b} \right)$. The persistent homology of $\mathcal{F} \left[ f \right]$ is defined the pair consisting of the family of homology groups associated with $\mathcal{F} \left[ f \right]$ and the induced maps describing how geometric features evolve along the filtration.

\begin{dfn}
  For real-valued function $f: X \to \mathbb{R}$ on a topological space $X$ and an integer $p \geq 0$, the persistent homology determined by sublevel set filtration of $f$ is
  \begin{align*}
    H_p \left( \mathcal{F} \left[ f \right] \right) = \left( \left( H_p \left( \Set{ f \leq a } \right) \right)_{a \in \mathbb{R}}, \left( \left( \iota_{a, b} \right)_* \right)_{a \leq b} \right).
  \end{align*}
  Also, $H_p \left( \mathcal{F} \left[ f \right] \right)$ is sometimes called the persistent homology of $f$.
\end{dfn}

Persistent homology, when expressed in this categorical form, is not immediately convenient for comparison or for stability analysis. Therefore, in practice, persistent homology is encoded as a multiset of points in $\bar{\mathbb{R}}^2$, called a persistence diagram.

A persistence diagram records, for each topological feature, the scale at which it appears and the scale at which it disappears. For example, suppose that in the family $ \left( H_p \left( \Set{ f \leq a } \right) \right)_{a \in \mathbb{R}}$, a homology class appears at $b \in \mathbb{R}$ and disappears at $d \in \mathbb{R}$. Then this feature is represented by a point $\left( b, d \right)$ in the diagram. If a homology class never vanishes, we record $d = \infty$. The persistence diagram is the multiset of all such points, together with all points on the diagonal $\Set{ \left( x, x \right) | x \in \bar{\mathbb{R}} }$

The above description is intended to provide conceptual intuition, but it leaves several mathematical issues implicit. In particular, the notions of ``appear'' and ``disappear'' are not formally defined in terms of the induced maps between homology groups. In rigorous treatments, persistence diagrams are defined uniquely via the decomposition theorem for persistence modules \cite{crawley2015decomposition, chazal2016structure}. For precise definitions, we refer the reader to \cite{dey2022computational} or \cite{chazal2016structure}.

In what follows, we denote by $\mathrm{dgm}_p~f$ the persistence diagram associated with the persistent homology $H_p \left( \mathcal{F} \left[ f \right] \right)$.

\subsubsection{Stability of persistence diagrams}

In our study, we investigate the Lipschitz continuity of persistence diagrams. To carry out such an analysis, it is essential that a metric structure be defined on the space of persistence diagrams. The bottleneck distance is one of the most widely used metrics for comparing persistence diagrams, and it is known to indeed define a distance between them.

\begin{dfn}
  Let $\mathrm{dgm}_p~f$ and $\mathrm{dgm}_p~g$ be two persistence diagrams. Let
  \begin{align*}
    \Pi = \Set{ \pi: \mathrm{dgm}_p~f \to \mathrm{dgm}_p~g }
  \end{align*}
  denote the set of all bijections from $\mathrm{dgm}_p~f$ to $\mathrm{dgm}_p~g$. The bottleneck distance between $\mathrm{dgm}_p~f$ and $\mathrm{dgm}_p~g$ is defined as
  \begin{align*}
    d_B \left( \mathrm{dgm}_p~f, \mathrm{dgm}_p~g \right) = \inf_{\pi \in \Pi} \sup_{x \in \mathrm{dgm}_p~f} \left\| x - \pi \left( x \right) \right\|_{\infty}.
  \end{align*}
\end{dfn}

Since persistent homology extracts topological features of data, it is expected to be stable under small perturbations of the input. This expectation is formalized in the following stability theorem, which states that the bottleneck distance between the resulting persistence diagrams is bounded above by the uniform norm of the difference between the input functions.

\begin{thm}[\cite{cohen2005stability, chazal2016structure}]\label{thm:stability}
  Let $f, g: X \to \mathbb{R}$ be real-valued function on a topological space $X$, and suppose that  $H_p \left( \mathcal{F} \left[ f \right] \right)$ and $H_p \left( \mathcal{F} \left[ g \right] \right)$ are $q$-tame. Then, we have
    \begin{align*}
    d_B \left( \mathrm{dgm}_p~f, \mathrm{dgm}_p~g \right) \leq \left\| f - g \right\|_\infty.
  \end{align*}
  Here, for real-valued function $f: X \to \mathbb{R}$ on $X$, $H_p \left( \mathcal{F} \left[ f \right] \right)$ is said to be $q$-tame if, for every real numbers $a \leq b$, the natural linear map $\left( \iota_{a, b} \right)_*: H_p \left( \Set{ f \leq a } \right) \to H_p \left( \Set{ f \leq b } \right)$ has finite rank.
\end{thm}

The following result is particularly useful for verifying the $q$-tameness condition in practical settings.
\begin{thm}[\cite{chazal2016structure}]\label{thm:q-tame-of-continuous}
  Let $X$ be a triangulable topological space and let $f: X \to \mathbb{R}$ be continuous. Then $H_p \left( \mathcal{F} \left[ f \right] \right)$ is $q$-tame.
\end{thm}
