\section{Example: Gray--Scott model}\label{sec:example}

In Section \ref{sec:method}, we discussed a general class of reaction-diffusion systems under conditions
\eqref{eq:reaction-lip}, \eqref{eq:reaction-bound}, and \eqref{eq:state-bound}. In this section, we show that the
Gray--Scott model, as a concrete reaction-diffusion system, satisfies conditions \eqref{eq:reaction-lip} and
\eqref{eq:reaction-bound}, and we investigate the Lipschitz continuity of persistence diagrams obtained from the
Turing patterns generated by the Gray--Scott model.

% Section \ref{sec:method}では一般の反応拡散方程式系に対して,条件\eqref{eq:reaction-lip},条件\eqref{eq:reaction-bound}および条件\eqref{eq:state-bound}を課して議論した。本節では,より具体的な反応拡散方程式系であるGray--Scottモデルが条件\eqref{eq:reaction-lip}および\eqref{eq:reaction-bound}を満たすこと示し,Gray--Scottモデルが生成するTuringパターンから得られるパーシステンス図のLipschitz連続性について検討する。

Let $d$ be a positive integer and set $\Omega = \left[ 0, 1 \right]^d$. Let $T > 0$ be a positive real number. Moreover, let $\Lambda \subset \left( 0, \infty \right)$, $\mathcal{F} \subset \left( 0, \infty \right)$, and $\mathcal{K} \subset \left( 0, \infty \right)$ be compact sets. As introduced in the Preliminary section, for a parameter $\theta = \left( F, k \right) \in \mathcal{F} \times \mathcal{K}$, the reaction term of the Gray--Scott model are given by
\begin{align*}
  R_1 \left( u_1, u_2; \theta \right) &= - u_1 u_2^2 + F \left( 1 - u_1 \right),\\
  R_2 \left( u_1, u_2; \theta \right) &= u_1 u_2^2 - \left( F + k \right) u_2,
\end{align*}
We consider the following two Gray--Scott models:
\begin{align}
  &\begin{dcases}\label{eq:general-gray-scott-normal}
      \dfrac{\partial u_1}{\partial t} \left( x, t \right) = \lambda_1 \Delta u_1 \left( x, t \right) + R_1 \left( u_1 \left( x, t \right), u_2 \left( x, t \right); \theta \right), & \left( x, t \right) \in \Omega \times \left( 0, T \right],\\
      \dfrac{\partial u_2}{\partial t} \left( x, t \right) = \lambda_2 \Delta u_2 \left( x, t \right) + R_2 \left( u_1 \left( x, t \right), u_2 \left( x, t \right); \theta \right), & \left( x, t \right) \in \Omega \times \left( 0, T \right],\\
      u_1 \left( x, 0 \right) = \tilde{u}_1 \left( x \right),~u_2 \left( x, 0 \right) = \tilde{u}_2 \left( x \right), & x \in \Omega\\
  \end{dcases}\\
  &\begin{dcases}\label{eq:general-gray-scott-prime}
      \dfrac{\partial u_1^\prime}{\partial t} \left( x, t \right) = \lambda_1^\prime \Delta u_1^\prime \left( x, t \right) + R_1 \left( u_1^\prime \left( x, t \right), u_2^\prime \left( x, t \right); \theta^\prime \right), & \left( x, t \right) \in \Omega \times \left( 0, T \right],\\
      \dfrac{\partial u_2^\prime}{\partial t} \left( x, t \right) = \lambda_2^\prime \Delta u_2^\prime \left( x, t \right) + R_2 \left( u_1^\prime \left( x, t \right), u_2^\prime \left( x, t \right); \theta^\prime \right), & \left( x, t \right) \in \Omega \times \left( 0, T \right], \\
      u_1^\prime \left( x, 0 \right) = \tilde{u}_1 \left( x \right),~u_2^\prime \left( x, 0 \right) = \tilde{u}_2 \left( x \right), & x \in \Omega\\
  \end{dcases}
\end{align}
Here, we set $D = \left( \lambda_1, \lambda_2 \right), D^\prime = \left( \lambda_1^\prime, \lambda_2^\prime \right)$ with $D, D^\prime \in \Lambda^2$, and $\theta = \left( F, k \right), \theta^\prime = \left( F^\prime, k^\prime \right)$ with $\theta, \theta^\prime \in \mathcal{F} \times \mathcal{K}$. We impose periodic boundary condition on $\Omega$. Furthermore, let $U = \left( u_1, u_2 \right)$ and $U^\prime = \left( u_1^\prime, u_2^\prime \right)$ be the solution vectors to \eqref{eq:general-gray-scott-normal} and \eqref{eq:general-gray-scott-prime}, respectively.

The following proposition corresponds to conditions \eqref{eq:reaction-lip} and \eqref{eq:reaction-bound}.
\begin{prop}\label{prop:gs-reaction-lip}
  Suppose that there exists a constant $L_4$, independent of $\theta$ and $\theta^\prime$, such that, for any $0 \leq t \leq T$,
  \begin{align*}
    \left\| U \left( \cdot, t \right) \right\|_\infty \leq L_4,~\left\| U^\prime \left( \cdot, t \right) \right\|_\infty \leq L_4
  \end{align*}
  Then, for any $i = 1, 2$ and $0 \leq t \leq T$, we have
  \begin{align*}
    &\left\| R_i \left( U \left( \cdot, t \right); \theta \right) - R_i \left( U^\prime \left( \cdot, t \right); \theta^\prime \right) \right\|_\infty\\
    &\hspace{0.5cm}\leq \left( 3L_4^2 + \sup_{\tau \in \mathcal{F}} \tau + \sup_{\tau \in \mathcal{K}} \tau \right) \left\| U \left( \cdot, t \right) - U^\prime \left( \cdot, t \right) \right\|_\infty + \left( 1 + 2L_4 \right) \left\| \theta - \theta^\prime \right\|_\infty.
  \end{align*}
  Moreover, we have
  \begin{align*}
    \left\|  R_i \left( U \left( \cdot, t \right); \theta \right) \right\|_\infty
    &\leq L_4^3 + \left( \sup_{\tau \in \mathcal{F}} \tau + \sup_{\tau \in \mathcal{K}} \tau \right) \left( 1 + L_4 \right).
  \end{align*}
\end{prop}

\begin{proof}
  We first consider $R_1$. By the triangle inequality,
  \begin{align*}
    &\left\| R_1 \left( U; \theta \right) - R_1 \left( U^\prime; \theta^\prime \right) \right\|_\infty\\
    &\hspace{0.5cm} = \left\| R_1 \left( U; \theta \right) - R_1 \left( U^\prime; \theta \right) + R_1 \left( U^\prime; \theta \right) - R_1 \left( U^\prime; \theta^\prime \right) \right\|_\infty\\
    &\hspace{0.5cm} \leq \left\| R_1 \left( U; \theta \right) - R_1 \left( U^\prime; \theta \right) \right\|_\infty
    + \left\| R_1 \left( U^\prime; \theta \right) - R_1 \left( U^\prime; \theta^\prime \right) \right\|_\infty.
  \end{align*}
  Here, we can obtain
  \begin{align*}
    \left\| R_1 \left( U; \theta \right) - R_1 \left( U^\prime; \theta \right) \right\|_\infty
    &\leq \left\| \left( - u_1 u_2^2 + F \left( 1 - u_1 \right) \right)
    - \left( - u_1^\prime {u_2^\prime}^2 + F \left( 1 - u_1^\prime \right) \right) \right\|_\infty \\
    &= \left\| \left( - u_1u_2^2 + u_1^\prime {u_2^\prime}^2 \right) + \left( F \left( 1 - u_1 \right) - F \left( 1 - u_1^\prime \right) \right) \right\|_\infty\\
    &\leq \left\| - u_1u_2^2 + u_1^\prime {u_2^\prime}^2 \right\|_\infty
    + \left\| F \left( 1 - u_1 \right) - F \left( 1 - u_1^\prime \right) \right\|_\infty\\
    &= \left\| - u_1 \left( u_2^2 - {u_2^\prime}^2 \right) - \left( u_1 - u_1^\prime \right) {u_2^\prime}^2 \right\|_\infty
    + F \left\| u_1 - u_1^\prime \right\|_\infty\\
    &\leq \left\| u_1 \right\|_\infty \left\| u_2^2 - {u_2^\prime}^2 \right\|_\infty + \left\| {u_2^\prime}^2  \right\|_\infty \left\| u_1 - u_1^\prime \right\|_\infty + F \left\| u_1 - u_1^\prime \right\|_\infty\\
    &= \left\| u_1 \right\|_\infty \left\| u_2 - u_2^\prime \right\|_\infty \left\| u_2 + u_2^\prime \right\|_\infty + \left\| {u_2^\prime}^2  \right\|_\infty \left\| u_1 - u_1^\prime \right\|_\infty + F \left\| u_1 - u_1^\prime \right\|_\infty\\
    &\leq L_4 \left\| U - U^\prime \right\|_\infty 2L_4 + L_4^2 \left\| U - U^\prime \right\|_\infty + F \left\| U - U^\prime \right\|_\infty\\
    &= \left( 3L_4^2 + F \right)  \left\| U - U^\prime \right\|_\infty\\
    &\leq \left( 3L_4^2 + \sup_{\tau \in \mathcal{F}} \tau \right)  \left\| U - U^\prime \right\|_\infty.
  \end{align*}
  Similarly, we have
  \begin{align*}
    \left\| R_1 \left( U^\prime; \theta \right) - R_1 \left( U^\prime; \theta^\prime \right) \right\|_\infty
    &\leq \left( 1 + L_4 \right) \left\| \theta - \theta^\prime \right\|_\infty.
  \end{align*}
  Therefore,
  \begin{align*}
    \left\| R_1 \left( U; \theta \right) - R_1 \left( U^\prime; \theta^\prime \right) \right\|_\infty \leq \left( 3L_4^2 + \sup_{\tau \in \mathcal{F}} \tau \right)  \left\| U - U^\prime \right\|_\infty + \left( 1 + L_4 \right) \left\| \theta - \theta^\prime \right\|_\infty
  \end{align*}
  By a similar computation for $R_2$, we can obtain
  \begin{align*}
    \left\| R_2 \left( U; \theta \right) - R_2 \left( U^\prime; \theta^\prime \right) \right\|_\infty \leq \left( 3L_4^2 + \sup_{\tau \in \mathcal{F}} \tau + \sup_{\tau \in \mathcal{K}} \tau \right)  \left\| U - U^\prime \right\|_\infty + 2L_4 \left\| \theta - \theta^\prime \right\|_\infty
  \end{align*}
  Combining these estimates, for $i = 1, 2$, we have
  \begin{align*}
    &\left\| R_i \left( U \left( \cdot, t \right); \theta \right) - R_i \left( U^\prime \left( \cdot, t \right); \theta^\prime \right) \right\|_\infty\\
    &\hspace{0.5cm}\leq \left( 3L_4^2 + \sup_{\tau \in \mathcal{F}} \tau + \sup_{\tau \in \mathcal{K}} \tau \right) \left\| U \left( \cdot, t \right) - U^\prime \left( \cdot, t \right) \right\|_\infty + \left( 1 + 2L_4 \right) \left\| \theta - \theta^\prime \right\|_\infty.
  \end{align*}

  Next, for $R_1$, we have
  \begin{align*}
    \left\| R_1 \left( u_1, u_2; \theta \right) \right\|_\infty
    &= \left\| - u_1 u_2^2 + F \left( 1 - u_1 \right) \right\|_\infty \\
    &\leq \left\| - u_1 u_2^2 \right\|_\infty + \left\| F \left( 1 - u_1 \right) \right\|_\infty\\
    &\leq L_4^3 + F \left( 1 + L_4 \right)\\
    &\leq L_4^3 + \sup_{\tau \in \mathcal{F}} \tau \left( 1 + L_4 \right)
  \end{align*}
  Similarly, for $R_2$, we have
  \begin{align*}
    \left\| R_2 \left( u_1, u_2; \theta \right) \right\|_\infty
    &\leq L_4^3 + \left( \sup_{\tau \in \mathcal{F}} \tau + \sup_{\tau \in \mathcal{K}} \tau \right) L_4
  \end{align*}
  Hence, for $i=1,2$, we can conclude
  \begin{align*}
    \left\|  R_i \left( U \left( \cdot, t \right); \theta \right) \right\|_\infty
    &\leq L_4^3 + \left( \sup_{\tau \in \mathcal{F}} \tau + \sup_{\tau \in \mathcal{K}} \tau \right) \left( 1 + L_4 \right).
  \end{align*}
\end{proof}

% $d$を正の整数とし,$\Omega = \left[ 0, 1 \right]^d$とおく。また,$T > 0$を正の実数とする。さらに,$\Lambda \subset \left( 0, \infty \right), \mathcal{F} \subset \left( 0, \infty \right), \mathcal{K} \subset \left( 0, \infty \right)$をそれぞれコンパクト集合とする。Preliminary sectionで紹介したように,パラメータ$\theta = \left( F, k \right) \in \mathcal{F} \times \mathcal{K}$に対するGray--Scottモデルの反応項は
% \begin{align*}
%   R_1 \left( u_1, u_2; \theta \right) &= - u_1 u_2^2 + F \left( 1 - u_1 \right),\\
%   R_2 \left( u_1, u_2; \theta \right) &= u_1 u_2^2 - \left( F + k \right) u_2,
% \end{align*}
% で与えられる。このとき,次の二つのGray--Scottモデルを考える。
% \begin{align}
%   &\begin{dcases}\label{eq:general-gray-scott-normal}
%       \dfrac{\partial u_1}{\partial t} \left( x, t \right) = \lambda_1 \Delta u_1 \left( x, t \right) + R_1 \left( u_1 \left( x, t \right), u_2 \left( x, t \right); \theta \right), & \left( x, t \right) \in \Omega \times \left( 0, T \right],\\
%       \dfrac{\partial u_2}{\partial t} \left( x, t \right) = \lambda_2 \Delta u_2 \left( x, t \right) + R_2 \left( u_1 \left( x, t \right), u_2 \left( x, t \right); \theta \right), & \left( x, t \right) \in \Omega \times \left( 0, T \right],\\
%       u_1 \left( x, 0 \right) = \tilde{u}_1 \left( x \right),~u_2 \left( x, 0 \right) = \tilde{u}_2 \left( x \right), & x \in \Omega\\
%   \end{dcases}\\
%   &\begin{dcases}\label{eq:general-gray-scott-prime}
%       \dfrac{\partial u_1^\prime}{\partial t} \left( x, t \right) = \lambda_1^\prime \Delta u_1^\prime \left( x, t \right) + R_1 \left( u_1^\prime \left( x, t \right), u_2^\prime \left( x, t \right); \theta^\prime \right), & \left( x, t \right) \in \Omega \times \left( 0, T \right],\\
%       \dfrac{\partial u_2^\prime}{\partial t} \left( x, t \right) = \lambda_2^\prime \Delta u_2^\prime \left( x, t \right) + R_2 \left( u_1^\prime \left( x, t \right), u_2^\prime \left( x, t \right); \theta^\prime \right), & \left( x, t \right) \in \Omega \times \left( 0, T \right], \\
%       u_1^\prime \left( x, 0 \right) = \tilde{u}_1 \left( x \right),~u_2^\prime \left( x, 0 \right) = \tilde{u}_2 \left( x \right), & x \in \Omega\\
%   \end{dcases}
% \end{align}
% ただし,$D = \left( \lambda_1, \lambda_2 \right), D^\prime = \left( \lambda_1^\prime, \lambda_2^\prime \right) \in \Lambda^2 $とし,$\theta = \left( F, k \right), \theta^\prime = \left( F^\prime, k^\prime \right) \in \mathcal{F} \times \mathcal{K}$とする。また,$\Omega$には周期境界条件を課すものとする。さらに,$U = \left( u_1, u_2 \right)$および$U^\prime = \left( u_1^\prime, u_2^\prime \right)$を\eqref{eq:general-gray-scott-normal}および\eqref{eq:general-gray-scott-prime}の解ベクトルとする。

% 次の命題は条件\eqref{eq:reaction-lip}および条件\eqref{eq:reaction-bound}に対応する。
% \begin{prop}\label{prop:gs-reaction-lip}
%   $\theta, \theta^\prime$に依存しないある定数$L_4$が存在して,任意の$0 < t \leq T$に対して,
%   \begin{align*}
%     \left\| U \left( \cdot, t \right) \right\|_\infty \leq L_4,~\left\| U^\prime \left( \cdot, t \right) \right\|_\infty \leq L_4
%   \end{align*}
%   が成り立つならば,任意の$i = 1, 2$および$0 < t \leq T$に対して,
%   \begin{align*}
%     &\left\| R_i \left( U \left( \cdot, t \right); \theta \right) - R_i \left( U^\prime \left( \cdot, t \right); \theta^\prime \right) \right\|_\infty\\
%     &\hspace{0.5cm}\leq \left( 3L_4^2 + \sup_{\tau \in \mathcal{F}} \tau + \sup_{\tau \in \mathcal{K}} \tau \right) \left\| U \left( \cdot, t \right) - U^\prime \left( \cdot, t \right) \right\|_\infty + \left( 1 + 2L_4 \right) \left\| \theta - \theta^\prime \right\|_\infty,
%   \end{align*}
%   が成り立つ。また,
%   \begin{align*}
%     \left\|  R_i \left( U \left( \cdot, t \right); \theta \right) \right\|_\infty
%     &\leq L_4^3 + \left( \sup_{\tau \in \mathcal{F}} \tau + \sup_{\tau \in \mathcal{K}} \tau \right) \left( 1 + L_4 \right),
%   \end{align*}
%   が成り立つ。
% \end{prop}

% \begin{proof}
%   まずは$R_1$について考えれば,
%   \begin{align*}
%     &\left\| R_1 \left( U; \theta \right) - R_1 \left( U^\prime; \theta^\prime \right) \right\|_\infty\\
%     &\hspace{0.5cm} = \left\| R_1 \left( U; \theta \right) - R_1 \left( U^\prime; \theta \right) + R_1 \left( U^\prime; \theta \right) - R_1 \left( U^\prime; \theta^\prime \right) \right\|_\infty\\
%     &\hspace{0.5cm} \leq \left\| R_1 \left( U; \theta \right) - R_1 \left( U^\prime; \theta \right) \right\|_\infty
%     + \left\| R_1 \left( U^\prime; \theta \right) - R_1 \left( U^\prime; \theta^\prime \right) \right\|_\infty
%   \end{align*}
%   となる。ここで,
%   \begin{align*}
%     \left\| R_1 \left( U; \theta \right) - R_1 \left( U^\prime; \theta \right) \right\|_\infty
%     &\leq \left\| \left( - u_1 u_2^2 + F \left( 1 - u_1 \right) \right)
%     - \left( - u_1^\prime {u_2^\prime}^2 + F \left( 1 - u_1^\prime \right) \right) \right\|_\infty \\
%     &= \left\| \left( - u_1u_2^2 + u_1^\prime {u_2^\prime}^2 \right) + \left( F \left( 1 - u_1 \right) - F \left( 1 - u_1^\prime \right) \right) \right\|_\infty\\
%     &\leq \left\| - u_1u_2^2 + u_1^\prime {u_2^\prime}^2 \right\|_\infty
%     + \left\| F \left( 1 - u_1 \right) - F \left( 1 - u_1^\prime \right) \right\|_\infty\\
%     &= \left\| - u_1 \left( u_2^2 - {u_2^\prime}^2 \right) - \left( u_1 - u_1^\prime \right) {u_2^\prime}^2 \right\|_\infty
%     + F \left\| u_1 - u_1^\prime \right\|_\infty\\
%     &\leq \left\| u_1 \right\|_\infty \left\| u_2^2 - {u_2^\prime}^2 \right\|_\infty + \left\| {u_2^\prime}^2  \right\|_\infty \left\| u_1 - u_1^\prime \right\|_\infty + F \left\| u_1 - u_1^\prime \right\|_\infty\\
%     &= \left\| u_1 \right\|_\infty \left\| u_2 - u_2^\prime \right\|_\infty \left\| u_2 + u_2^\prime \right\|_\infty + \left\| {u_2^\prime}^2  \right\|_\infty \left\| u_1 - u_1^\prime \right\|_\infty + F \left\| u_1 - u_1^\prime \right\|_\infty\\
%     &\leq L_4 \left\| U - U^\prime \right\|_\infty 2L_4 + L_4^2 \left\| U - U^\prime \right\|_\infty + F \left\| U - U^\prime \right\|_\infty\\
%     &= \left( 3L_4^2 + F \right)  \left\| U - U^\prime \right\|_\infty\\
%     &\leq \left( 3L_4^2 + \sup_{\tau \in \mathcal{F}} \tau \right)  \left\| U - U^\prime \right\|_\infty
%   \end{align*}
%   が得られる。同様にして,
%   \begin{align*}
%     \left\| R_1 \left( U^\prime; \theta \right) - R_1 \left( U^\prime; \theta^\prime \right) \right\|_\infty
%     &\leq \left( 1 + L_4 \right) \left\| \theta - \theta^\prime \right\|_\infty 
%   \end{align*}
%   が得られる。よって,
%   \begin{align*}
%     \left\| R_1 \left( U; \theta \right) - R_1 \left( U^\prime; \theta^\prime \right) \right\|_\infty \leq \left( 3L_4^2 + \sup_{\tau \in \mathcal{F}} \tau \right)  \left\| U - U^\prime \right\|_\infty + \left( 1 + L_4 \right) \left\| \theta - \theta^\prime \right\|_\infty
%   \end{align*}
%   となる。$R_2$についても同様の計算を行えば,
%   \begin{align*}
%     \left\| R_2 \left( U; \theta \right) - R_2 \left( U^\prime; \theta^\prime \right) \right\|_\infty \leq \left( 3L_4^2 + \sup_{\tau \in \mathcal{F}} \tau + \sup_{\tau \in \mathcal{K}} \tau \right)  \left\| U - U^\prime \right\|_\infty + 2L_4 \left\| \theta - \theta^\prime \right\|_\infty
%   \end{align*}
%   となる。以上のことから,$i = 1, 2$に対して,
%   \begin{align*}
%     &\left\| R_i \left( U \left( \cdot, t \right); \theta \right) - R_i \left( U^\prime \left( \cdot, t \right); \theta^\prime \right) \right\|_\infty\\
%     &\hspace{0.5cm}\leq \left( 3L_4^2 + \sup_{\tau \in \mathcal{F}} \tau + \sup_{\tau \in \mathcal{K}} \tau \right) \left\| U \left( \cdot, t \right) - U^\prime \left( \cdot, t \right) \right\|_\infty + \left( 1 + 2L_4 \right) \left\| \theta - \theta^\prime \right\|_\infty,
%   \end{align*}
%   が成り立つ。

%   次に$R_1$について,
%   \begin{align*}
%     \left\| R_1 \left( u_1, u_2; \theta \right) \right\|_\infty
%     &= \left\| - u_1 u_2^2 + F \left( 1 - u_1 \right) \right\|_\infty \\
%     &\leq \left\| - u_1 u_2^2 \right\|_\infty + \left\| F \left( 1 - u_1 \right) \right\|_\infty\\
%     &\leq L_4^3 + F \left( 1 + L_4 \right)\\
%     &\leq L_4^3 + \sup_{\tau \in \mathcal{F}} \tau \left( 1 + L_4 \right)
%   \end{align*}
%   が得られる。$R_2$についても同様に計算して,
%   \begin{align*}
%     \left\| R_2 \left( u_1, u_2; \theta \right) \right\|_\infty
%     &\leq L_4^3 + \left( \sup_{\tau \in \mathcal{F}} \tau + \sup_{\tau \in \mathcal{K}} \tau \right) L_4
%   \end{align*}
%   となる。以上のことから,$i = 1, 2$に対して,
%   \begin{align*}
%     \left\|  R_i \left( U \left( \cdot, t \right); \theta \right) \right\|_\infty
%     &\leq L_4^3 + \left( \sup_{\tau \in \mathcal{F}} \tau + \sup_{\tau \in \mathcal{K}} \tau \right) \left( 1 + L_4 \right),
%   \end{align*}
%   が成り立つ。
% \end{proof}

% Gray--Scottモデルの解ベクトルの上界$L_4$が存在する状況では,
% \begin{align*}
%   M_1 &= \dfrac{ d }{ \inf_{\tau \in \Lambda} \tau } \left( L_4 + L_4^3 + \left( \sup_{\tau \in \mathcal{F}} \tau + \sup_{\tau \in \mathcal{K}} \tau \right) \left( 1 + L_4 \right) \right) \exp \left( \left( 3L_4^2 + \sup_{\tau \in \mathcal{F}} \tau + \sup_{\tau \in \mathcal{K}} \tau \right) t \right),\\
%   M_2 &=  \left( 1 + 2L_4 \right) t \exp \left( \left( 3L_4^2 + \sup_{\tau \in \mathcal{F}} \tau + \sup_{\tau \in \mathcal{K}} \tau \right) t \right),
% \end{align*}
% と定めることで,Proposition \ref{prop:gs-reaction-lip}とTheorem \ref{thm:main-theorem}により $i = 1, 2$に対して,
% \begin{align*}
%   d_B \left( \mathrm{dgm}_p~u_i \left( \cdot, t \right), \mathrm{dgm}_p~u_i^\prime \left( \cdot, t \right) \right) \leq M_1 \left\| D - D^\prime \right\|_\infty + M_2 \left\| \theta - \theta^\prime \right\|_\infty.
% \end{align*}
% が成り立つ。このような方法により,Section \ref{sec:method}の議論はGray--Scottモデルに応用できることがわかる。

In a setting where an upper bound $L_4$ exists for the solution vector of the Gray--Scott model, define
\begin{align*}
  M_1 \left( t \right) &= \dfrac{ d }{ \inf_{\tau \in \Lambda} \tau } \left( L_4 + L_4^3 + \left( \sup_{\tau \in \mathcal{F}} \tau + \sup_{\tau \in \mathcal{K}} \tau \right) \left( 1 + L_4 \right) \right) \exp \left( \left( 3L_4^2 + \sup_{\tau \in \mathcal{F}} \tau + \sup_{\tau \in \mathcal{K}} \tau \right) t \right),\\
  M_2 \left( t \right) &=  \left( 1 + 2L_4 \right) t \exp \left( \left( 3L_4^2 + \sup_{\tau \in \mathcal{F}} \tau + \sup_{\tau \in \mathcal{K}} \tau \right) t \right),
\end{align*}
then, by Proposition \ref{prop:gs-reaction-lip} and Theorem \ref{thm:main-theorem}, for $i = 1, 2$, we have
\begin{align}\label{eq:main-theorem-gs}
  d_B \left( \mathrm{dgm}_p~u_i \left( \cdot, t \right), \mathrm{dgm}_p~u_i^\prime \left( \cdot, t \right) \right) \leq M_1 \left\| D - D^\prime \right\|_\infty + M_2 \left\| \theta - \theta^\prime \right\|_\infty.
\end{align}
Using this approach, we see that the discussion in Section \ref{sec:method} can be applied to the Gray--Scott model.
