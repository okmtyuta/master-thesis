\section{Conclusion}\label{sec:conclusion}

% 本論文では,Turingパターンから得られるパーシステンス図の間のボトルネック距離が,Turingパターンを生成する反応拡散方程式系のモデルパラメータに対して Lipschitz 連続となるための条件を検討した。半群法による解の評価とパーシステントホモロジーの安定性定理を組み合わせることで,条件 \eqref{eq:reaction-lip},\eqref{eq:reaction-bound},および \eqref{eq:state-bound} が成り立つならば,パーシステンス図がモデルパラメータに関して Lipschitz 連続であることを示した。さらに,本研究の議論が具体的な反応拡散方程式系であるGray--Scottモデルに適用できることも示した。

% 一方で,数値実験の結果からもわかるように,本研究の結果から得られる不等式\eqref{eq:inequality-main-theorem}は,非常に緩い評価になっており実用的とは言えない。

% また,本研究で用いた解析手法は,反応拡散方程式系の解の評価とパーシステントホモロジーの安定性定理を直接的に組み合わせたものであり,「かたち」を捉えるというパーシステントホモロジー本来の特徴を十分に活用できているとは言い難い。本研究では,モデルパラメータが十分に近い場合におけるパーシステンス図の類似性を検討したが,パラメータ空間上では必ずしも近くないにもかかわらず,トポロジカルな観点からは類似したパターンが得られるような状況を見いだすことができれば,Turingパターンの「かたち」に着目した新たな理解につながると期待される。このような観点からの解析も,今後の研究課題である。

In this paper, we investigated conditions under which the bottleneck distance between persistence diagrams obtained from Turing patterns is Lipschitz continuous with respect to the model parameters of the reaction-diffusion systems that generate those patterns. By combining solution estimates via semigroup methods with the stability theorem of persistent homology, we showed that if conditions \eqref{eq:reaction-lip}, \eqref{eq:reaction-bound}, and \eqref{eq:state-bound} are satisfied, then the persistence diagrams are Lipschitz continuous with respect to the model parameters. We also showed that the arguments developed in this study can be applied to the Gray--Scott model, a concrete example of reaction-diffusion equations.

At the same time, the following issues remain as important topics for future research.
\begin{enumerate}
  \item From inequality \eqref{eq:inequality-main-theorem} presented in Section \ref{sec:method}, we see that if the uniform norm of the model parameters of two reaction-diffusion systems is sufficiently small, then the bottleneck distance between the corresponding two persistence diagrams can also be small. However, as indicated by the numerical simulation results in Section \ref{sec:simulation-study}, the situations in which this estimate is useful are limited. In particular, since the coefficients $M_1 \left( t \right)$ and $M_2 \left( t \right)$ grow exponentially in $t$, inequality \eqref{eq:inequality-main-theorem} becomes almost meaningless once sufficient time has elapsed. Therefore, constructing a tighter estimate than inequality \eqref{eq:inequality-main-theorem} is an important direction for future research.
  \item In Section \ref{sec:example}, the conclusion is derived under the assumption that the uniform norm of the solution vector is bounded by a constant $L_4$. By making use of a comparison theorem, it may be possible to construct this constant $L_4$ explicitly, thereby eliminating the unclear assumption in Section \ref{sec:example}. This examination is also important.
  \item In this study, we restricted the domain on which the reaction-diffusion system is defined to $\Omega = \left[ 0, 1 \right]^d$ and assumed periodic boundary conditions only. However, the domain in which reaction-diffusion phenomena arise in practice is not necessarily a rectangular one, and the geometry of the domain as well as the boundary conditions can strongly affect the selection of emerging patterns. Therefore, extending $\Omega$ to a more general domain and considering more general boundary conditions, including Dirichlet and Neumann boundary conditions are important directions for future research in order to broaden the applicability of our approach.
  \item Our theoretical analysis was conducted on a finite time interval. In the context of pattern formation, however, it is often important to investigate long-time behavior (e.g., steady states). Developing a framework that enables theoretical estimates of the bottleneck distance on an infinite time interval remains an important challenge.
  % \item 本研究においては,安定性定理を利用して,反応拡散系の解の類似性からパーシステンス図の類似性を評価した。したがって,反応拡散系の解が類似していないときのパーシステンス図の類似性については何も議論できていない。しかし,関数として類似していないにも関わらずパターンは類似しているためにパーシステンス図が類似している状況を解析することも重要である。この状況を調べるためには研究の枠組みを大きく変える必要があるが,パターン解析におけるパーシステントホモロジーの活用のためには重要な課題である。
  \item In this study, we used a stability theorem to evaluate the similarity of persistence diagrams via the similarity of solutions to a reaction-diffusion system. Therefore, we have not been able to discuss the similarity of persistence diagrams when the solutions to the reaction-diffusion system are not similar. However, it is also important to analyze situations in which persistence diagrams are similar because the patterns are similar, even though the functions themselves are not similar. Investigating this situation would require a major change in the research framework, but it is an important issue for leveraging persistent homology in pattern analysis.
\end{enumerate}

% 次に,Section \ref{sec:example}では,解ベクトルの一様ノルムが定数$L_4$で抑えられるという仮定の下で結論を導いている。比較原理を活用することで,比較定理を活用すれば,この定数$L_4$を明示的に構成できる可能性があり,Section \ref{sec:example}における不明瞭な仮定を排除できると考えられる。比較定理を利用した$L_4$のinformalな構成はAppendix Bで行っている。


% An informal construction of $L_4$ based on the comparison theorem is given in Appendix B.

% 本研究では,反応拡散方程式系を定義する領域を$\Omega = \left[ 0, 1 \right]^d$に限定し,境界条件として周期境界条件のみを仮定した。しかし,実際の反応拡散現象が生じる領域は必ずしも直方体領域とは限らず,また領域形状や境界条件は解の安定性や現れるパターンの選択に大きく影響するため,対応するパーシステンス図の性質も変化し得る。したがって,$\Omega$をより一般の領域へ拡張するとともに,境界条件についてもDirichlet条件やNeumann条件を扱うことは,本手法の適用範囲を広げる上で重要な研究課題である。この点については,Appendix Cで予備的な考察を行う。


% We provide a preliminary discussion of this point in Appendix C.


% On the other hand, our results are restricted to situations on a finite time interval $\left[ 0, T \right]$ Consequently, they cannot be directly applied to states in which sufficient time has elapsed and the pattern formation has reached a stable regime. Since many studies on Turing patterns, such as \cite{spector2024persistent}, focus on stable patterns, the analysis in the infinite-time limit remains an important topic for future research.

% In addition, the analytical approach employed in this study directly combines solution estimates for reaction-diffusion systems with the stability theorem of persistent homology, and thus does not fully exploit the intrinsic ability of persistent homology to capture “shape.” In this work, we examined the similarity of persistence diagrams when the model parameters are sufficiently close. However, if one could identify situations in which patterns are topologically similar despite the corresponding parameters not being close in the parameter space, this would lead to a deeper understanding of the “shape” of Turing patterns. Investigating such scenarios is another direction for future research.