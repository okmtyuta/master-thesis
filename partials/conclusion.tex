\section{Conclusion}

% 本論文では,Turingパターンから得られるパーシステンス図の間のボトルネック距離が,Turingパターンを生成する反応拡散方程式系のモデルパラメータに対して Lipschitz 連続となるための条件を検討した。半群法による解の評価とパーシステントホモロジーの安定性定理を組み合わせることで,条件 \eqref{eq:reaction-lip},\eqref{eq:reaction-bound},および \eqref{eq:state-bound} が成り立つならば,パーシステンス図がモデルパラメータに関して Lipschitz 連続であることを示した。さらに,いくつかのパラメータ設定に対して数値実験を行い,理論解析の結果を視覚的に検証した。

% 一方で,本研究の結果は,有限時間区間$\left[ 0, T \right]$上に限定された状況におけるものである。そのため,十分に時間が経過し,パターン形成が安定した状態に対しては,本結果を直接適用することはできない。Turingパターンの解析においては,安定したパターンを対象とする研究も多く行われていることから\cite{spector2024persistent},無限時間極限における解析は今後の重要な研究課題である。

% また,本研究で用いた解析手法は,反応拡散方程式系の解の評価とパーシステントホモロジーの安定性定理を直接的に組み合わせたものであり,「かたち」を捉えるというパーシステントホモロジー本来の特徴を十分に活用できているとは言い難い。本研究では,モデルパラメータが十分に近い場合におけるパーシステンス図の類似性を検討したが,パラメータ空間上では必ずしも近くないにもかかわらず,トポロジカルな観点からは類似したパターンが得られるような状況を見いだすことができれば,Turingパターンの「かたち」に着目した新たな理解につながると期待される。このような観点からの解析も,今後の研究課題である。

In this paper, we investigated conditions under which the bottleneck distance between persistence diagrams obtained from Turing patterns is Lipschitz continuous with respect to the model parameters of the reaction--diffusion systems that generate those patterns. By combining solution estimates based on semigroup methods with the stability theorem of persistent homology, we showed that if conditions \eqref{eq:reaction-lip}, \eqref{eq:reaction-bound}, and \eqref{eq:state-bound} are satisfied, then the persistence diagrams are Lipschitz continuous with respect to the model parameters. Furthermore, we conducted numerical experiments for several parameter settings to visually validate the results of the theoretical analysis.

On the other hand, our results are restricted to situations on a finite time interval $\left[ 0, T \right]$ Consequently, they cannot be directly applied to states in which sufficient time has elapsed and the pattern formation has reached a stable regime. Since many studies on Turing patterns, such as \cite{spector2024persistent}, focus on stable patterns, the analysis in the infinite-time limit remains an important topic for future research.

In addition, the analytical approach employed in this study directly combines solution estimates for reaction-diffusion systems with the stability theorem of persistent homology, and thus does not fully exploit the intrinsic ability of persistent homology to capture “shape.” In this work, we examined the similarity of persistence diagrams when the model parameters are sufficiently close. However, if one could identify situations in which patterns are topologically similar despite the corresponding parameters not being close in the parameter space, this would lead to a deeper understanding of the “shape” of Turing patterns. Investigating such scenarios is another direction for future research.