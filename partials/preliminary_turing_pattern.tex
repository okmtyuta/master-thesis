\subsection{Turing pattern}

\subsubsection{Turing pattern and reaction-diffusion systems}

% In 1952, A. M. Turing mathematically showed that, under idealized conditions, hypothetical chemical substances known as morphogens can mutually regulate each other's production, thereby forming characteristic spatial patterns in their concentration distributions \cite{turing1990chemical}. Such phenomena are generally modeled by reaction-diffusion systems of the form
% \begin{align}
%   &\begin{dcases}\label{eq:general-reaction-diffusion-equations}
%     \dfrac{\partial u_i}{\partial t} = \lambda_i \Delta u_i + R_i \left( u_1 , \dots, u_n \right), \\
%     1 \leq i \leq n,\\
%   \end{dcases}
% \end{align}
% where $u_i$ denotes the concentration of each morphogen, and $\lambda_i$ is the diffusion coefficient of $u_i$, which quantifies the strength of its diffusion. In this system, the term $\lambda_i \Delta u_i \left( x, t \right)$ in \eqref{eq:general-reaction-diffusion-equations} describes diffusive transport of morphogens, while $R_i \left( u_1 \left( x, t \right), \dots, u_n \left( x, t \right) \right)$ represents their chemical reactions. Turing patterns are spatially heterogeneous patterns that arise in reaction-diffusion systems via diffusion-driven instability of a homogeneous steady state.

In 1952, A. M. Turing mathematically showed, under idealized conditions, that spatial patterns can spontaneously emerge from a homogeneous state through the interaction and diffusion of hypothetical chemical substances known as morphogens \cite{turing1990chemical}. Such phenomena are commonly modeled by the following reaction-diffusion system:
\begin{align}
  &\begin{dcases}\label{eq:general-reaction-diffusion-equations}
    \dfrac{\partial u_i}{\partial t} = \lambda_i \Delta u_i + R_i \left( u_1 , \dots, u_n \right), \\
    1 \leq i \leq n,
  \end{dcases}
\end{align}
where $u_i$ denotes the concentration of the $i$-th morphogen, $\lambda_i$ is the diffusion coefficient of the $i$-th morphogen $u_i$, quantifying the strength of its diffusion, and $\Delta$ is Laplace operator with respect to the spatial variable $x = \left( x_1, \dots, x_n \right) \in \mathbb{R}^d$, defined for a sufficiently smooth function $f: \mathbb{R}^d \to \mathbb{R}$ by
\begin{align*}
  \Delta f \left( x \right) = \sum_{j = 1}^d \dfrac{\partial^2 f}{\partial x_j^2} \left( x \right).
\end{align*}
In \eqref{eq:general-reaction-diffusion-equations}, the term $\lambda_i \Delta u_i(x,t)$ describes diffusive transport, while $R_i \left(u_1(x,t),\dots,u_n(x,t)\right)$ represents the chemical reactions. Spatially heterogeneous patterns that arise in reaction-diffusion systems via diffusion-driven instability of a homogeneous steady state are called Turing patterns.

Whether biological body-surface patterns can indeed be explained by Turing patterns had long been a matter of debate. In 1995, however, Kondo and Asai demonstrated that developmental changes in the skin pattern of Pomacanthus fish can be explained by a Turing pattern mechanism \cite{kondo1995reaction}. This study supported Turing patterns as an important mathematical model for understanding the emergence of biological body patterns.

Turing patterns are now known to account for a wide variety of biological surface patterns beyond those of Pomacanthus. For example, several studies have reproduced the skin patterns of zebras, leopards, jaguars, and pufferfish using Turing-type reaction-diffusion models \cite{gravan2004evolving, liu2006two, de2020leopard}. 

\subsubsection{Gray--Scott model}

% P. Gray and S. K. Scott analyzed autocatalytic processes in continuous flow stirred tank reactor (CSTR) systems in \cite{gray1983autocatalytic, gray1984autocatalytic}. This autocatalytic mechanism, with reactants $U_1$ and $U_2$ and product $P$, can be described by the following chemical reactions:
% \begin{align}
%   U_1 + 2U_2 &\to 3U_2, \label{eq:gs-mutula-reaction} \\
%   U_2 &\to P. \label{eq:gs-productive-reaction}
% \end{align}
% Here, the volume of the reactor is assumed to be constant and equal to $V$. We assume that a reactant $U_1$ with concentration $c$ flows into the reactor at a volumetric flow rate $q$ per unit time. Let $k_1$ and $k_2$ denote the reaction rate constants of reactions \eqref{eq:gs-mutula-reaction} and \eqref{eq:gs-productive-reaction}, respectively, and let $u_1^\star$ and $u_2^\star$ denote the concentrations of reactants $U_1$ and $U_2$ in the reactor, respectively. Then, including diffusion, the time evolution of $u_1^\star$ and $u_2^\star$ with respect to time $t^\star$ in the reactor is given by
% \begin{align*}
%   \begin{dcases}
%     \dfrac{\partial u_1^\star}{\partial t^\star} = \lambda_1^\star \Delta u_1^\star - k_1 {u_1^\star}{u_2^\star}^2 + \dfrac{q}{V} \left( c - u_1^\star \right) ,\\
%     \dfrac{\partial u_2^\star}{\partial t^\star} = \lambda_2^\star \Delta u_2^\star + k_1 {u_1^\star}{u_2^\star}^2 - k_2 u_2^\star + \dfrac{q}{V} \left( 0 - u_2^\star \right).
%   \end{dcases}
% \end{align*}
% Here, introducing the nondimensional variables and parameters
% \begin{align*}
% u_1 = \dfrac{u_1^\star}{c}, ~
% u_2 = \dfrac{u_2^\star}{c}, ~
% t = k_1 c^2 t^\star, ~
% F = \dfrac{q / V}{k_1 c^2}, ~
% k = \dfrac{k_2}{k_1 c^2}, ~
% \lambda_1 = \dfrac{\lambda_1^\star}{k_1 c^2}, ~
% \lambda_2 = \dfrac{\lambda_2^\star}{k_1 c^2},
% \end{align*}
% then we obtain
% \begin{align}
%   \begin{dcases}\label{eq:general-gray-scott}
%   \dfrac{\partial u_1}{\partial t} = \lambda_1 \Delta u_1 - u_1 u_2^2 + F \left( 1 - u_1 \right),\\
%   \dfrac{\partial u_2}{\partial t} = \lambda_2 \Delta u_2 + u_1 u_2^2 - \left( F + k \right) u_2.
%   \end{dcases}
% \end{align}

% The system \eqref{eq:general-gray-scott} is known as the Gray--Scott model, a prototypical reaction-diffusion system that generates Turing-type patterns. Pearson \cite{pearson1993complex} demonstrated that the Gray--Scott model produces an extremely rich variety of spatiotemporal patterns depending on the choice of parameters $\left( F, k \right)$ Examples of Turing patterns generated by the Gray-Scott model are shown in Figure~XXX. These patterns offer important insights into the role of Turing mechanisms in biological morphogenesis.

P.~Gray and S.~K.~Scott analyzed autocatalytic processes in continuous-flow stirred-tank reactor (CSTR) systems \cite{gray1983autocatalytic,gray1984autocatalytic}. The autocatalytic mechanism, considered by Gray and Scott, involving reactants $U_1$ and $U_2$ and an inert product $P$ is described by the reactions:
\begin{align}
  U_1 + 2U_2 &\to 3U_2, \label{eq:gs-autocatalytic-reaction}\\
  U_2 &\to P. \label{eq:gs-decay-reaction}
\end{align}
Here, assume that the reactor volume is constant and equal to $V$, and that $U_1$ is supplied to the reactor at concentration $c$ with volumetric flow rate $q$. Let $k_1$ and $k_2$ denote the reaction rate constants in \eqref{eq:gs-autocatalytic-reaction} and \eqref{eq:gs-decay-reaction}, respectively, and let $u_1^\star$ and $u_2^\star$ denote the concentrations of $U_1$ and $U_2$. Under a stirred-tank assumption, there is no spatial variation in concentration. Hence, the evolution of $u_1^\star$ and $u_2^\star$ is given by
\begin{align*}
  \begin{dcases}
    \dfrac{\partial u_1^\star}{\partial t^\star} = - k_1 u_1^\star (u_2^\star)^2 + \dfrac{q}{V}\left(c-u_1^\star\right),\\
    \dfrac{\partial u_2^\star}{\partial t^\star} = k_1 u_1^\star (u_2^\star)^2 - k_2 u_2^\star - \dfrac{q}{V}u_2^\star .
  \end{dcases}
\end{align*}
On the basis of studies by Pearson \cite{pearson1993complex} and others, this model was generalized to the following form, incorporating spatial concentration variations due to diffusion:
\begin{align}
  \begin{dcases}\label{eq:gs-dimensional}
    \dfrac{\partial u_1^\star}{\partial t^\star} = \lambda_1^\star \Delta u_1^\star - k_1 u_1^\star (u_2^\star)^2 + \dfrac{q}{V}\left(c-u_1^\star\right),\\
    \dfrac{\partial u_2^\star}{\partial t^\star} = \lambda_2^\star \Delta u_2^\star + k_1 u_1^\star (u_2^\star)^2 - k_2 u_2^\star - \dfrac{q}{V}u_2^\star .
  \end{dcases}
\end{align}
Here $\lambda_i^\star$ are diffusion coefficients. We nondimensionalize the system by introducing
\begin{align*}
  u_1 = \dfrac{u_1^\star}{c},~
  u_2 = \dfrac{u_2^\star}{c},~
  t = k_1 c^2 t^\star,~
  x = \dfrac{x^\star}{L},
\end{align*}
and the dimensionless parameters
\begin{align*}
  F = \dfrac{q/V}{k_1 c^2},~
  k = \dfrac{k_2}{k_1 c^2},~
  \lambda_1 = \dfrac{\lambda_1^\star}{k_1 c^2 L^2},~
  \lambda_2 = \dfrac{\lambda_2^\star}{k_1 c^2 L^2},
\end{align*}
where $L$ is a characteristic length scale. 
Then \eqref{eq:gs-dimensional} becomes
\begin{align}
  \begin{dcases}\label{eq:general-gray-scott}
    \dfrac{\partial u_1}{\partial t} = \lambda_1 \Delta u_1 - u_1 u_2^2 + F(1-u_1),\\
    \dfrac{\partial u_2}{\partial t} = \lambda_2 \Delta u_2 + u_1 u_2^2 - (F+k)u_2.
  \end{dcases}
\end{align}

The system \eqref{eq:general-gray-scott} is known as the Gray--Scott model, a prototypical reaction-diffusion system that exhibits Turing-type pattern formation. 
Pearson \cite{pearson1993complex} showed that the Gray--Scott model produces a rich variety of spatiotemporal patterns depending on the parameter choice $(F,k)$. These patterns provide useful insights into diffusion-driven pattern formation in reaction-diffusion systems.

In this study, we analyze the Gray--Scott model in Section \ref{sec:example} as a concrete example of a reaction-diffusion system that generates Turing patterns.
