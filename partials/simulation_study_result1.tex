\begin{figure}[h]
  \begin{minipage}{0.36\linewidth}
    \centering
    \includegraphics[width=\linewidth]{figures/u_tilde.png}
    \subcaption{$\tilde{u}_1 \left( \left( x_1, x_2 \right) \right)$}\label{fig:u-tilde-1}
  \end{minipage}
  \hfill
  \begin{minipage}{0.43\linewidth}
    \centering
    \includegraphics[width=\linewidth]{figures/v_tilde.png}
    \subcaption{$\tilde{u_2} \left( \left( x_1, x_2 \right) \right)$}\label{fig:u-tilde-2}
  \end{minipage}

  \caption{$\tilde{u}_1$ and $\tilde{u}_2$ used as the initial condition.}
  \label{fig:initial-condition}
\end{figure}

\begin{figure}[h]
  \begin{minipage}{0.24\linewidth}
    \centering
    \includegraphics[width=\linewidth]{figures/u_theta1_t=3000.png}
    \subcaption{$u_{1, \theta_1} \left( \left( x_1, x_2 \right), 3000 \right)$}\label{fig:u-theta1}
  \end{minipage}
  \hfill
  \begin{minipage}{0.24\linewidth}
    \centering
    \includegraphics[width=\linewidth]{figures/u_theta1_prime_t=3000.png}
    \subcaption{$u_{1, \theta_1^\prime} \left( \left( x_1, x_2 \right), 3000 \right)$}\label{fig:u-theta1-prime}
  \end{minipage}
  \hfill
  \begin{minipage}{0.24\linewidth}
    \centering
    \includegraphics[width=\linewidth]{figures/u_theta2_t=3000.png}
    \subcaption{$u_{1, \theta_2} \left( \left( x_1, x_2 \right), 3000 \right)$}\label{fig:u-theta2}
  \end{minipage}
  \hfill
  \begin{minipage}{0.24\linewidth}
    \centering
    \includegraphics[width=\linewidth]{figures/u_theta2_prime_t=3000.png}
    \subcaption{$u_{1, \theta_2^\prime} \left( \left( x_1, x_2 \right), 3000 \right)$}\label{fig:u-theta2-prime}
  \end{minipage}

  \caption{Figures of the patterns generated by the parameter settings $\theta \in \Set{ \theta_1, \theta_1^\prime, \theta_2, \theta_2^\prime }$. The plots show $u_{1,\theta}$ at $t = 3000$.}
  \label{fig:u-theta}
\end{figure}

\begin{figure}[h]
  \begin{minipage}{0.32\linewidth}
    \centering
    \includegraphics[width=\linewidth]{figures/exp1_result_theta1_theta1_prime.png}
    \subcaption{$\theta_1$ and $\theta_1^\prime$}\label{fig:theta1-theta-1-prime}
  \end{minipage}
  \hfill
  \begin{minipage}{0.32\linewidth}
    \centering
    \includegraphics[width=\linewidth]{figures/exp1_result_theta2_theta2_prime.png}
    \subcaption{$\theta_2$ and $\theta_2^\prime$}\label{fig:theta2-theta2-prime}
  \end{minipage}
  \hfill
  \begin{minipage}{0.32\linewidth}
    \centering
    \includegraphics[width=\linewidth]{figures/exp1_result_theta1_theta2.png}
    \subcaption{$\theta_1$ and $\theta_2$}\label{fig:theta1-theta2}
  \end{minipage}

  \caption{Simulation results for inequality \eqref{eq:main-theorem-gs} for the parameter pairs $\left( \mu, \nu \right) \in \Set{ \left( \theta_1, \theta_1^\prime \right), \left( \theta_2, \theta_2^\prime \right), \left( \theta_1, \theta_2 \right) }$. The horizontal axis represents time, and the vertical axis shows each quantity on a logarithmic scale. Moreover, to avoid divergence of the logarithm near $x=0$, we added a small value $\varepsilon = 10^{-24}$ and computed $\log \left( x + \varepsilon \right)$.}
  \label{fig:simulation-for-inequality}

\end{figure}
