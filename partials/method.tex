\section{Method}

本セクションでは,Introductionで述べたように,Turingパターンから得られるパーシステンス図のLipschitz連続性について検討する。まずは,パラメータ付けられた一般の反応拡散方程式系を定義して,その方程式系に対して証明したい定理を述べる。そして,Preliminaryで準備した内容をもとにして定理の証明を行う。

\subsection{Problem setting}

$n, d, p$をそれぞれ正の整数とし,$T > 0$を正の実数とする。また,$\Omega = \left[ 0, 1 \right]^d$とする。さらに,$\Lambda$および$\Theta$をそれぞれ$\left( 0, \infty \right)$および$\left( 0, \infty \right)^p$上のコンパクト集合とする。このとき,次の二つの反応拡散方程式系を考える。
\begin{align}
  &\begin{dcases}\label{eq:reaction-diffusion-equations-normal}
    \dfrac{\partial u_i}{\partial t} \left( x, t \right) = \lambda_i \Delta u_i \left( x, t \right) + R_i \left( u_1 \left( x, t \right), \dots, u_n \left( x, t \right); \theta \right), & \left( x, t \right) \in \Omega \times \left( 0, T \right] \\
    u_i \left( x, 0 \right) = \tilde{u}_i \left( x \right), & x \in \Omega\\
    1 \leq i \leq n,\\
  \end{dcases}\\
  &\begin{dcases}\label{eq:reaction-diffusion-equations-prime}
    \dfrac{\partial u_i^\prime}{\partial t} \left( x, t \right) = \lambda_i^\prime \Delta u_i^\prime \left( x, t \right) + R_i \left( u_1^\prime \left( x, t \right), \dots, u_n^\prime \left( x, t \right); \theta^\prime \right), & \left( x, t \right) \in \Omega \times \left( 0, T \right] \\
    u_i^\prime \left( x, 0 \right) = \tilde{u}_i \left( x \right), & x \in \Omega\\
    1 \leq i \leq n,\\
  \end{dcases}
\end{align}
ただし,$D = \left( \lambda_1, \dots, \lambda_n \right), D^\prime = \left( \lambda_1^\prime, \dots, \lambda_n^\prime \right) \in \Lambda^n$および$\theta, \theta^\prime \in \Theta$は反応拡散方程式系のパラメータであり,$\Omega$の境界条件としては周期境界条件を考える。また,\eqref{eq:reaction-diffusion-equations-normal}および\eqref{eq:reaction-diffusion-equations-prime}の解からなるベクトルをそれぞれ$U = \left( u_1, \dots, u_n \right)$および$U^\prime = \left( u_1^\prime, \dots, u_n^\prime \right)$とする。

このような設定のもとで,$1 \leq i \leq n$に対して,$u_i$および$u_i^\prime$から得られる$q$-パーシステンス図$\mathrm{dgm_q}~u_i$および$\mathrm{dgm_q}~u_i^\prime$の間のボトルネック距離$d_B \left( \mathrm{dgm_q}~u_i, \mathrm{dgm_q}~u_i^\prime \right)$が,モデルパラメータ$D, \theta$および$D^\prime, \theta^\prime$によってどのように評価されるかを調べる。具体的には,次の定理に示すようなLipschitz連続性を証明する。

\begin{thm}\label{thm:main-theorem}
  $1 \leq i \leq n$とし,$q$を正の整数とする。$\theta, \theta^\prime$に依存しない定数$L_1, L_2, L_3, L_4 \in \mathbb{R}$が存在して,任意の$x \in \Omega$と$0 < t \leq T$に対して,
  \begin{align}
    \left\| R_i \left( U \left( x, t \right); \theta \right) - R_i \left( U^\prime \left( x, t \right); \theta^\prime \right) \right\|_\infty
    &\leq L_1 \left\| \left( U \left( x, t \right) \right)- \left( U^\prime \left( x, t \right) \right)  \right\|_\infty
    + L_2 \left\| \theta - \theta^\prime \right\|_\infty\\
    \left\| R_i \left( U \left( x, t \right); \theta \right) \right\|_\infty
    &\leq L_3,
    ~\left\| R_i \left( U^\prime \left( x, t \right); \theta^\prime \right) \right\|_\infty
    \leq L_3,\\
    \left\| U \left( x, t \right) \right\|_\infty
    &\leq L_4,
    ~\left\| U^\prime \left( x, t \right) \right\|_\infty
    \leq L_4,
  \end{align}
  が成り立つならば,$\theta, \theta^\prime$に依存しない定数$M_1, M_2 \in \mathbb{R}$が存在して,任意の$x \in \Omega$と$0 < t \leq T$に対して,
  \begin{align*}
    d_B \left( \mathrm{dgm}_q u_i \left( \cdot, t \right), \mathrm{dgm}_q u_i^\prime \left( \cdot, t \right) \right) \leq M_1 \left\| D - D^\prime \right\|_\infty + M_2 \left\| \theta - \theta^\prime \right\|_\infty
  \end{align*}
  が成り立つ。
\end{thm}

この定理は,\eqref{eq:reaction-diffusion-equations-normal}および\eqref{eq:reaction-diffusion-equations-prime}のそれぞれのモデルパラメータの差分$\left\| D - D^\prime \right\|_\infty$および$\left\| \theta - \theta^\prime \right\|_\infty$が十分に小さければ,ボトルネック距離$d_B \left( \mathrm{dgm_q}~u_i, \mathrm{dgm_q}~u_i^\prime \right)$も十分に小さいことを示している。

\subsection{Bounds for solution differences under parameter perturbations}

Theorem \ref{thm:main-theorem}の証明は,Theorem xxxを利用して行う。Theorem xxxを利用するためには,$\left\| u_i \left( \cdot, t \right) - u_i^\prime \left( \cdot, t \right) \right\|_\infty$の評価が必要である。そのため,まずは$\left\| u_i \left( \cdot, t \right) - u_i^\prime \left( \cdot, t \right) \right\|_\infty$から始める。

Preliminaryの議論により,\eqref{eq:reaction-diffusion-equations-normal}および\eqref{eq:reaction-diffusion-equations-prime}の古典解は,
\begin{align*}
  u_i \left( \cdot, t \right) &= H_{\lambda_i} \left( t \right) \tilde{u}_i \left( \cdot \right) + \int_0^t H_{\lambda_i} \left( t - s \right) R_i \left( u_1 \left( \cdot, s \right), \dots, u_n \left( \cdot, s \right); \theta \right) ds, \\
  u_i^\prime \left( \cdot, t \right) &= H_{\lambda_i^\prime} \left( t \right) \tilde{u}_i \left( \cdot \right) + \int_0^t H_{\lambda_i^\prime} \left( t - s \right) R_i \left( u_1 \left( \cdot, s \right), \dots, u_n \left( \cdot, s \right); \theta^\prime \right) ds,
\end{align*}
で与えられるから,
\begin{align*}
  \left\| u_i \left( \cdot, t \right) - u_i^\prime \left( \cdot, t \right) \right\|_\infty
  &\leq \left\| \left( H_{\lambda_i} \left( t \right) - H_{\lambda_i^\prime} \left( t \right) \right) \tilde{u}_i \left( \cdot \right) \right\|_\infty\\
  &\hspace{1cm} + \left\| \int_0^t \left( H_{\lambda_i} \left( t - s \right) R_i \left( U \left( \cdot, t \right); \theta \right) - H_{\lambda_i^\prime} \left( t - s \right) R_i \left( U \left( \cdot, t \right); \theta^\prime \right) \right) \right\|_\infty\\
  &\leq \left\| \left( H_{\lambda_i} \left( t \right) - H_{\lambda_i^\prime} \left( t \right) \right) \tilde{u}_i \left( \cdot \right) \right\|_\infty \\
  &\hspace{1cm} + \left\| \int_0^t \left( H_{\lambda_i} \left( t - s \right) - H_{\lambda_i^\prime} \left( t - s \right) \right) R_i \left( U \left( \cdot, t \right); \theta \right) ds \right\|_\infty \\
  &\hspace{1cm} + \left\| \int_0^t H_{\lambda_i} \left( t - s \right) \left( R_i \left( U \left( \cdot, t \right); \theta \right) - R_i \left( U^\prime \left( \cdot, t \right); \theta^\prime \right) \right) ds \right\|_\infty
\end{align*}
となる。したがって,
\begin{align*}
  E_1 &= \left\| \left( H_{\lambda_i} \left( t \right) - H_{\lambda_i^\prime} \left( t \right) \right) \tilde{u}_i \left( \cdot \right) \right\|_\infty, \\
  E_2 &= \left\| \int_0^t \left( H_{\lambda_i} \left( t - s \right) - H_{\lambda_i^\prime} \left( t - s \right) \right) R_i \left( U \left( \cdot, t \right); \theta \right) ds \right\|_\infty, \\
  E_3 &= \left\| \int_0^t H_{\lambda_i} \left( t - s \right) \left( R_i \left( U \left( \cdot, t \right); \theta \right) - R_i \left( U^\prime \left( \cdot, t \right); \theta^\prime \right) \right) ds \right\|_\infty,
\end{align*}
とおけば,$E_1, E_2, E_3$の評価を通して,$\left\| u_i \left( \cdot, t \right) - u_i^\prime \left( \cdot, t \right) \right\|_\infty$の評価が得られる。そこで以下では,$E_1, E_2, E_3$の評価を行っていく。

\begin{lem}
  $\lambda \in \Lambda$および$0 < t \leq T$とする。このとき,任意の$u \in C_{\mathrm{per}} \left( \Omega \right)$に対して,
  \begin{align*}
    \left\| H_{\lambda} \left( t \right) u \right\|_\infty \leq  \left\| u \right\|_\infty
  \end{align*}
  が成り立つ。
\end{lem}

\begin{proof}
  $H_{\lambda} \left( t \right)$の定義により,$0 < t \leq T$ならば
  \begin{align*}
    \left\| H_{\lambda} \left( t \right) u \right\|_\infty
    & \leq \left\| K_{\lambda, t} * u \right\|_{\infty}\\
    & \leq \left\| K_{\lambda, t} \right\|_{L_1} \left\| u \right\|_{\infty}\\
    &= \left\| u \right\|_{\infty}
  \end{align*}
  を得る。
\end{proof}

\begin{lem}
  任意の$0 < t \leq T$および$\lambda, \lambda^\prime \in \Lambda$に対して,
  \begin{align*}
    \left\| K_{\lambda, t} - K_{\lambda^\prime, t} \right\|_{L_1} \leq \dfrac{d}{\inf_{\tau \in \Lambda} \tau} \left| \lambda - \lambda^\prime \right|
  \end{align*}
  が成り立つ。さらに,$u \in C_{\mathrm{per}} \left( \Omega \right)$に対して,
  \begin{align*}
    \left\| \left( H_{\lambda} \left( t \right) - H_{\lambda^\prime} \left( t \right) \right) u \right\|_{\infty} \leq \dfrac{d}{\inf_{\tau \in \Lambda} \tau} \left| \lambda - \lambda^\prime \right| \left\| u \right\|_\infty
  \end{align*}
  が成り立つ。
\end{lem}

\begin{proof}
  任意の$0 < t \leq T$と$x \in \mathbb{R}^d$
  \begin{align*}
    \dfrac{\partial}{\partial \lambda} g_{\lambda, t} \left( x \right)
    &= - \dfrac{d}{2} \dfrac{1}{\left( 4 \pi \lambda t \right)^{d / 2 + 1}} 4 \pi t \exp \left( - \dfrac{ \left\| x \right\|^2 }{ 4 \lambda t } \right) + \dfrac{1}{\left( 4 \pi \lambda t \right)^{d / 2}} \exp \left( - \dfrac{ \left\| x \right\|^2 }{ 4 \lambda t } \right) \dfrac{ \left\| x \right\|^2 }{ 4 \lambda^2 t }\\
    &= \left( - \dfrac{ d }{ 2 \lambda } + \dfrac{ \left\| x \right\|^2 }{ 4 \lambda^2 t } \right) g_{\lambda, t} \left( x \right)
  \end{align*}
  となるから,
  \begin{align*}
    \left\| \dfrac{\partial}{\partial \lambda} K_{\lambda, t} \right\|_{L^1}
    &= \left\| \dfrac{\partial}{\partial \lambda} \sum_{k \in \mathbb{Z}^d} g_{\lambda, t} \left( x + k \right) \right\|_{L^1}\\
    &= \left\| \sum_{k \in \mathbb{Z}^d} \dfrac{\partial}{\partial \lambda} g_{\lambda, t} \left( x + k \right) \right\|_{L^1}\\
    & \leq \sum_{k \in \mathbb{Z}^d} \left\| \dfrac{\partial}{\partial \lambda} g_{\lambda, t} \left( x + k \right) \right\|_{L^1}\\
    &= \sum_{k \in \mathbb{Z}^d} \int_{\left[ 0, 1 \right]^d} \left| \dfrac{\partial}{\partial \lambda} g_{\lambda, t} \left( x + k \right) \right| dx\\
    &= \int_{\mathbb{R}^d} \left| \dfrac{\partial}{\partial \lambda} g_{\lambda, t} \left( x \right) \right| dx\\
    &\leq \int_{\mathbb{R}^d} \left( \dfrac{d}{2 \lambda} + \dfrac{\left\| x \right\|^2}{4 \lambda^2 t} \right) g_{\lambda, t} \left( x \right) dx\\
    &= \dfrac{d}{2 \lambda} \int_{\mathbb{R}^d} g_{\lambda, t} \left( x \right) dx + \dfrac{1}{4 \lambda^2 t} \int_{\mathbb{R}^d} \left\| x \right\|^2 g_{\lambda, t} \left( x \right) dx\\
    &= \dfrac{d}{2 \lambda} + \dfrac{1}{4 \lambda^2 t} 2d \lambda t\\
    &= \dfrac{d}{\lambda}
  \end{align*}
  となる。次に,任意の$\lambda, \lambda^\prime \in \Lambda$に対して,
  \begin{align*}
    K_{\lambda, t} \left( x \right) - K_{\lambda^\prime, t} \left( x \right)
    &= \sum_{k \in \mathbb{Z}^d} \left( g_{\lambda, t} \left( x + k \right) - g_{\lambda^\prime, t} \left( x + k \right) \right)\\
    &= \sum_{k \in \mathbb{Z}^d} \int_{\lambda^\prime}^{\lambda} \dfrac{\partial}{\partial \tau} g_{\tau, t} \left( x + k \right) d \tau \\
    &= \int_{\lambda^\prime}^\lambda \sum_{k \in \mathbb{Z}^d} \dfrac{\partial}{\partial \tau} g_{\tau, t} \left( x + k \right) d \tau\\
    &= \int_{\lambda^\prime}^\lambda \dfrac{\partial}{\partial \tau} K_{\tau, t} \left( x \right) d \tau
  \end{align*}
  となるから,
  \begin{align*}
    \left\| K_{\lambda, t} - K_{\lambda^\prime, t} \right\|_{L^1}
    &= \int_{\left[ 0, 1 \right]^d} \left| \int_{\lambda^\prime}^\lambda \dfrac{\partial}{\partial \tau} K_{\tau, t} \left( x \right) d \tau \right| dx\\
    &\leq \int_{\left[ 0, 1 \right]^d} \int_{\lambda^\prime}^\lambda \left| \dfrac{\partial}{\partial \tau} K_{\tau, t} \left( x \right) \right| d \tau dx\\
    &= \int_{\lambda^\prime}^\lambda \left\| \dfrac{\partial}{\partial \tau} K_{\tau, t} \left( x \right) \right\|_{L^1} d\tau\\
    &\leq \sup_{\tau \in \mathcal{\lambda}} \left\| \dfrac{\partial}{\partial \tau} K_{\tau, t} \left( x \right)  \right\|_{L_1} \left| \lambda - \lambda^\prime \right|\\
    &\leq \sup_{\tau \in \Lambda} \dfrac{d}{\tau} \left| \lambda - 
    \lambda^\prime \right|\\
    &= \dfrac{d}{\inf_{\tau \in \Lambda} \tau} \left| \lambda - \lambda^\prime \right|
  \end{align*}
  を得る。最後に,
  \begin{align*}
    \left\| \left( H_{\lambda} \left( t \right) - H_{\lambda^\prime} \left( t \right) \right) u \right\|_{\infty}
    &\leq \left\| K_{\lambda, t} - K_{\lambda^\prime, t} \right\|_{L^1} \left\| u \right\|_\infty\\
    &\leq \dfrac{d}{\inf_{\tau \in \Lambda} \tau} \left| \lambda - \lambda^\prime \right| \left\| u \right\|_\infty
  \end{align*}
  を得る。
\end{proof}

% \begin{lem}
%   $\theta, \theta^\prime$に依存しない定数$M_1, M_2 \in \mathbb{R}$が存在して,任意の$1 \leq i \leq n$および$x \in \Omega$と$0 < t \leq T$に対して,
%   \begin{align*}
%     \left\| u_i \left( x, t \right) - u_i^\prime \left( x, t \right) \right\|_\infty \leq M_1 \left\| \mathcal{D} - \mathcal{D}^\prime \right\|_\infty + M_2 \left\| \theta - \theta^\prime \right\|
%   \end{align*}
%   が成り立つ。
% \end{lem}

% \begin{proof}
%   Theorem XXXによって,任意の$\left( x, t \right) \in \Omega \times \left[ 0, T \right)$に対して,
%   \begin{align*}
%     \left\| u_i \left( x, t \right) - u_i^\prime \left( x, t \right) \right\|_\infty
%     &\leq \left\| \left( e^{t D_i \Delta} - e^{t D_i^\prime \Delta} \right) \tilde{u}_i \left( x \right) \right\|_\infty\\
%     &\hspace{1cm} + \left\| \int_0^t \left( e^{\left( t - s \right) D_i \Delta} R_i \left( U \left( x, t \right); \theta \right) - e^{\left( t - s \right) D_i^\prime \Delta} R_i \left( U \left( x, t \right); \theta^\prime \right) \right) \right\|_\infty\\
%     &\leq \left\| \left( e^{t D_i \Delta} - e^{t D_i^\prime \Delta} \right) \tilde{u}_i \left( x \right) \right\|_\infty\\
%     &\hspace{1cm} + \left\| \int_0^t \left( e^{\left(t - s\right)D_i\Delta} - e^{\left(t - s\right)D_i^\prime\Delta} \right) R_i \left( U \left( x, t \right); \theta \right) ds \right\|_\infty \\
%     &\hspace{1cm} + \left\| \int_0^t e^{\left( t - s \right)D_i^\prime\Delta} \left( R_i \left( U \left( x, t \right); \theta \right) - R_i \left( U^\prime \left( x, t \right); \theta^\prime \right) \right) ds \right\|_\infty
%   \end{align*}
%   となる。まず,第1項について,
%   \begin{align*}
%     \left\| \left( e^{t D_i \Delta} - e^{t D_i^\prime \Delta} \right) \tilde{u}_i \right\|_\infty
%     &\leq \dfrac{d}{\inf \mathcal{D}} \left| D_i - D_i^\prime \right| \left\| \tilde{u}_i \right\|_\infty\\
%     &\leq \dfrac{d}{\inf \mathcal{D}} \left| D_i - D_i^\prime \right| L_4
%   \end{align*}
%   となる。次に,第2項について,
%   \begin{align*}
%     \left\| \int_0^t \left( e^{\left(t - s\right)D_i\Delta} - e^{\left(t - s\right)D_i^\prime\Delta} \right) R_i \left( U \left( x, t \right); \theta \right) ds \right\|_\infty
%     &\leq \int_0^t \left\| \left( e^{\left(t - s\right)D_i\Delta} - e^{\left(t - s\right)D_i^\prime\Delta} \right) R_i \left( U \left( x, t \right); \theta \right) \right\|_\infty ds\\
%     &\leq \int_0^t \dfrac{d}{\inf \mathcal{D}} \left| D_i - D_i^\prime \right| \left\| R_i \left( U \left( x, t \right); \theta \right) \right\|_\infty ds\\
%     &\leq \int_0^t \dfrac{d}{\inf \mathcal{D}} \left| D_i - D_i^\prime \right| L_3 ds\\
%     &\leq T \left| D_i - D_i^\prime \right| L_3 \\
%   \end{align*}
%   となる。最後に,第3項について,
%   \begin{align*}
%     &\left\| \int_0^t e^{\left( t - s \right)D_i^\prime\Delta} \left( R_i \left( U \left( x, t \right); \theta \right) - R_i \left( U^\prime \left( x, t \right); \theta^\prime \right) \right) ds \right\|_\infty\\
%     &\hspace{2cm}\leq \int_0^t \left\|  e^{\left( t - s \right)D_i^\prime\Delta} \left( R_i \left( U \left( x, t \right); \theta \right) - R_i \left( U^\prime \left( x, t \right); \theta^\prime \right) \right) \right\|_\infty ds\\
%     &\hspace{2cm}\leq \int_0^t \left\| R_i \left( U \left( x, t \right); \theta \right) - R_i \left( U^\prime \left( x, t \right); \theta^\prime \right) \right\|_\infty ds\\
%     &\hspace{2cm}\leq \int_0^t \left( L_1 \left\| U \left( x, t \right)  - U^\prime \left( x, t \right) \right\|_\infty + L_2 \left\| \theta - \theta^\prime \right\|_\infty \right) ds\\
%     &\hspace{2cm}\leq L_1 \int_0^t \left\| U \left( x, t \right)  - U^\prime \left( x, t \right) \right\|_\infty ds + T L_2 \left\| \theta - \theta^\prime \right\|_\infty
%   \end{align*}
%   となる。したがって,
%   \begin{align*}
%     \left\| u_i \left( x, t \right) - u_i^\prime \left( x, t \right) \right\|_\infty
%     &\leq \dfrac{d}{\inf \mathcal{D}} \left| D_i - D_i^\prime \right| L_4
%     + T \left| D_i - D_i^\prime \right| L_3 + T L_2 \left\| \theta - \theta^\prime \right\|_\infty\\
%     &\hspace{1cm} + L_1 \int_0^t \left\| U \left( x, t \right)  - U^\prime \left( x, t \right) \right\|_\infty ds\\
%     &\leq \left( \dfrac{d}{\inf \mathcal{D}} + TL_3 \right) \left| D_i - D_i^\prime \right| L_4 + T L_2 \left\| \theta - \theta^\prime \right\|_\infty\\
%     &\hspace{1cm} + L_1 \int_0^t \left\| U \left( x, t \right)  - U^\prime \left( x, t \right) \right\|_\infty ds
%   \end{align*}
%   となるから,
%   \begin{align*}
%     \left\| U \left( x, t \right)  - U^\prime \left( x, t \right) \right\|_\infty
%     &\leq \left( \dfrac{d}{\inf \mathcal{D}} + TL_3 \right) \left\| \mathbf{D} - \mathbf{D}^\prime \right\|_\infty L_4 + n T L_2 \left\| \theta - \theta^\prime \right\|_\infty\\
%     &\hspace{1cm} + n L_1 \int_0^t \left\| U \left( x, t \right)  - U^\prime \left( x, t \right) \right\|_\infty ds
%   \end{align*}
%   ここにGronwallの不等式を用いれば,
%   \begin{align*}
%     \left\| U \left( x, t \right)  - U^\prime \left( x, t \right) \right\|_\infty
%     &\leq \left[ \left( \dfrac{d}{\inf \mathcal{D}} + TL_3 \right) L_4 \left\| \mathbf{D} - \mathbf{D}^\prime \right\|_\infty + n T L_2 \left\| \theta - \theta^\prime \right\|_\infty \right] e^{n L_1 T}
%   \end{align*}
%   を得る。ここで,
%   \begin{align*}
%     M_1 &= \left( \dfrac{d}{\inf \mathcal{D}} + TL_3 \right) L_4 e^{n L_1 T},\\
%     M_2 &= n T L_2 e^{n L_1 T}
%   \end{align*}
%   とおくことにより,
%   \begin{align*}
%     \left\| u_i \left( x, t \right) - u_i^\prime \left( x, t \right) \right\|_\infty \leq M_1 \left\| \mathbf{D} - \mathbf{D}^\prime \right\|_\infty + M_2 \left\| \theta - \theta^\prime \right\|
%   \end{align*}
%   を得る。
% \end{proof}
