\subsection{Semigroup method}

In this study, to apply the stability theorem of persistent homology, it is necessary to obtain appropriate estimates for classical solutions of reaction-diffusion systems. However, classical solutions of such systems are generally difficult to express in closed form, and the presence of nonlinear terms makes direct analysis complicated. To address this issue, we introduce the semigroup approach, which enables us to separate the linear and nonlinear components and rewrite the system into a formulation that is more amenable to estimation.

% In this subsection, we outline the semigroup framework for reaction-diffusion systems, which governs the linear part of the reaction-diffusion system. This enables us to treat explicitly the time evolution arising from the diffusive term, thereby making it easier to combine with the analysis of the nonlinear reaction term. For more detailed theoretical background on semigroup methods, see \cite{pazy2012semigroups}.

In this subsection, we provide an overview of how the semigroup method can be applied to reaction-diffusion systems. In particular, we explain how the semigroup framework allows one to represent classical solutions of reaction-diffusion systems. A fully rigorous introduction to the semigroup method would require defining concepts that are unnecessarily technical for the purposes of this study. Therefore, we restrict ourselves to the minimum material needed here. More detailed discussions are given in Appendix~\ref{sec:supplementary-explanation-of-semigroup-method}. For a rigorous and comprehensive treatment, see \cite{pazy2012semigroups}.

We first introduce several pieces of notation. Let $d$ be a positive integer. Also, Let $\Omega = \left[ 0, 1 \right]^d$ and let $C_{\mathrm{per}} \left( \Omega \right)$ be the space of all continuous real-valued periodic functions having period $1$. Then, $\left( C_{\mathrm{per}} \left( \Omega \right), \left\| \cdot \right\|_{\infty} \right)$ is a Banach space. We define the subset $C^2_{\mathrm{per}} \left( \Omega \right)$ of $C_{\mathrm{per}} \left( \Omega \right)$ by
\begin{align*}
  C^2_{\mathrm{per}} \left( \Omega \right) = \Set{ u \in C_{\mathrm{per}} \left( \Omega \right) | \dfrac{\partial u}{\partial x_j}, \dfrac{\partial^2 u}{\partial x_j^2} \in C_{\mathrm{per}} \left( \Omega \right), 1 \leq j \leq d }.
\end{align*}
Then, we define the operator $\Delta$ on $C^2_{\mathrm{per}} \left( \Omega \right)$ for all $u \in C^2_{\mathrm{per}} \left( \Omega \right)$,
\begin{align*}
  \Delta u = \sum_{j = 1}^d \dfrac{\partial^2 u}{\partial x_j^2}.
\end{align*}
For all $\lambda > 0$ and $t > 0$, define
\begin{align*}
  g_{\lambda, t} \left( x \right) &= \dfrac{1}{\left( 4 \pi \lambda t \right)^{d / 2}} \exp \left( - \dfrac{ \left\| x \right\|^2 }{ 4 \lambda t } \right),\\
  K_{\lambda, t} \left( x \right) &= \sum_{k \in \mathbb{Z}^d} g_{\lambda ,t} \left( x + k \right).
\end{align*}
Then, for $t \geq 0$, define for any $f \in C_{\mathrm{per}} \left( \Omega \right)$ and $x \in \Omega$
\begin{align}
  \left( H_{\lambda} \left( t \right) f \right) \left( x \right) =
  \begin{dcases}
    \left( K_{\lambda, t} * f \right) \left( x \right) = \int_{\Omega} K_{\lambda, t} \left( x - y \right) f \left( y \right) dy, & t > 0, \\
    f \left( x \right), & t = 0.
  \end{dcases}
\end{align}
The family $\mathcal{H}_\lambda = \Set{ H_\lambda \left( t \right) }_{t \geq 0}$ is known as $C_0$-semigroup on $C_{\mathrm{per}} \left( \Omega \right)$ generated by $\lambda \Delta$. Here, for the definitions of the $C_0$-semigroup, see the Appendix \ref{sec:supplementary-explanation-of-semigroup-method}.

Next, we investigate an application of the $C_0$-semigroup $\mathcal{H}_\lambda$ to reaction-diffusion systems. Let $n$ be a positive integers and let $T > 0$ be a positive real number. Under this setting, as described in \eqref{eq:general-reaction-diffusion-equations}, a system of reaction-diffusion systems is generally given by
\begin{align}\label{eq:reaction-diffusion-systems-at-pre-semigroup}
  \begin{dcases}
    \dfrac{\partial u_i}{\partial t} \left( x, t \right) = \lambda_i \Delta u_i \left( x, t \right) + R_i \left( u_1 \left( x, t \right), \dots, u_n \left( x, t \right) \right), & \left( x, t \right) \in \Omega \times \left( 0, T \right], \\
    u_i \left( x, 0 \right) = \tilde{u}_i \left( x \right), & x \in \Omega,\\
    1 \leq i \leq n. &
  \end{dcases}
\end{align}
Here we impose periodic boundary conditions on $\Omega$. By the discussion in the Appendix \ref{sec:supplementary-explanation-of-semigroup-method}, the following theorem holds.
\begin{thm}\label{thm:mild-solution-of-reaction-diffusion-at-pre-semigroup}
  When the reaction term $R_i$ satisfies appropriate conditions, then the classical solution $u_i$ of \eqref{eq:reaction-diffusion-systems-at-pre-semigroup} satisfies, for any $0 \leq t \leq T$, the following integral equation:
  \begin{align}\label{eq:mild-solution-of-reaction-diffusion-at-pre-semigroup}
    u_i \left( \cdot, t \right) = H_{\lambda_i} \left( t \right) \tilde{u}_i \left( \cdot \right) + \int_0^t H_{\lambda_i} \left( t - s \right) R_i \left( u_1 \left( \cdot, s \right), \dots, u_n \left( \cdot, s \right) \right) ds.
  \end{align}
\end{thm}
By this theorem, whenever the reaction-diffusion systems \eqref{eq:reaction-diffusion-systems-at-pre-semigroup} admits a classical solution, that solution necessarily satisfies the formula \eqref{eq:mild-solution-of-reaction-diffusion-at-pre-semigroup}. Therefore, when estimating the classical solution of reaction-diffusion systems, it suffices to work with the representation \eqref{eq:mild-solution-of-reaction-diffusion-at-pre-semigroup}. In the subsequent Method section, we will use the representation of the classical solution given by Theorem \ref{thm:mild-solution-of-reaction-diffusion-at-pre-semigroup} to develop estimates for the classical solution.
