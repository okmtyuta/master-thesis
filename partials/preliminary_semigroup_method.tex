\subsection{Semigroup method}

\subsubsection{Abstract semigroup framework}

In this study, to apply the stability theorem of persistent homology, it is necessary to obtain appropriate estimates for classical solutions of reaction-diffusion systems. However, classical solutions of such systems are generally difficult to express in closed form, and the presence of nonlinear terms makes direct analysis complicated. To address this issue, we introduce the semigroup approach, which enables us to separate the linear and nonlinear components and rewrite the system into a formulation that is more amenable to estimation.

In this subsection, we outline the basic framework of the semigroup method employed in this work and examine the $C_0$-semigroup generated by the Laplace operator, which governs the linear part of the reaction-diffusion system. This allows us to treat explicitly the time evolution arising from the diffusive term, thereby facilitating its combination with the analysis of the nonlinear reaction term. For more detailed theoretical background on semigroup methods, we can refer the reader to \cite{pazy2012semigroups}.

In what follows, let $\left( B, \left\| \cdot \right\|_B \right)$ be a Banach space, and let $\mathrm{id}_B$ denote the identity operator on $B$. We also fix a positive integer $d$. We begin by defining semigroups and $C_0$-semigroups on Banach spaces, together with their generators.

\begin{dfn}
  A family of bounded linear operators $\mathcal{S} = \Set{ S \left( t \right): B \to B }_{t \geq 0}$ is called a semigroup on $B$ if it satisfies the following conditions:
  \begin{enumerate}
    \item $S \left( 0 \right) = \mathrm{id}_B$.
    \item For all $s, t \geq 0$, $S \left( s + t \right) = S \left( s \right) S \left( t \right)$.
  \end{enumerate}
\end{dfn}

\begin{dfn}
  Let $\mathcal{S} = \Set{ S \left( t \right) }_{t \geq 0}$ be a semigroup on $B$. If, for all $x \in B$,
  \begin{align*}
    \lim_{t \searrow 0} S \left( t \right) x = x, 
  \end{align*}
  then $\mathcal{S}$ is called a $C_0$-semigroup on $B$.
\end{dfn}

\begin{dfn}
  Let $\mathcal{S} = \Set{ S \left( t \right) }_{t \geq 0}$ be a $C_0$-semigroup on $B$. For all $x \in B$, define the operator $A$ by
  \begin{align*}
    Ax = \lim_{t \searrow 0} \dfrac{ S \left( t \right) x - x }{t},
  \end{align*}
  where $A$ is defined on the domain
  \begin{align*}
    D \left( A \right) = \Set{ x \in B | \lim_{t \searrow 0} \dfrac{ S \left( t \right) x - x }{t} \in B }.
  \end{align*}
  Then $A$ is called the infinitesimal generator of $\mathcal{S}$.
\end{dfn}

Next, we introduce the representation of classical solutions of differential equations obtained via the semigroup method. Let $\mathcal{S} = \Set{ S \left( t \right) }_{t \geq 0}$ be a $C_0$-semigroup on $B$, and let $A$ be the infinitesimal generator of $\mathcal{S}$. Let $R: B \to B$ be a mapping on $B$. 
We consider the abstract initial value problem for $u(t) \in D(A)$:
\begin{align}\label{eq:preliminary-semigroup-initial-value-problem}
  \begin{dcases}
    \dfrac{d u}{dt} \left( t \right) =  A u \left( t \right) + R \left( u \left( t \right) \right), & t > 0,\\
    u \left( 0 \right) = \tilde{u}.
  \end{dcases}
\end{align}

The following theorem shows that any classical solution of this abstract Cauchy problem can be represented by an integral equation involving the semigroup $\mathcal{S}$.
\begin{thm}
  If there exists a constant $C \in \mathbb{R}$ such that for all $x, y \in B$,
  \begin{align*}
    \left\| R \left( x \right) - R \left( y \right) \right\|_B \leq C \left\| x - y \right\|_B,
  \end{align*}
  then the classical solution $u \left( t \right) \in D \left( A \right)$ of \eqref{eq:preliminary-semigroup-initial-value-problem} satisfies the integral equation
  \begin{align}\label{eq:rep-classical-solution-by-semigroup}
    u \left( t \right) = S \left( t \right) \tilde{u} + \int_0^t S \left( t - s \right) R \left( u \left( s \right) \right) ds.
  \end{align}
\end{thm}
By this theorem, whenever the initial value problem \eqref{eq:preliminary-semigroup-initial-value-problem} admits a classical solution, that solution necessarily satisfies the formula \eqref{eq:rep-classical-solution-by-semigroup}. Therefore, when estimating the classical solution, it suffices to work with the representation \eqref{eq:rep-classical-solution-by-semigroup}.


\subsubsection{Heat semigroup generated by the Laplacian}

To analyze reaction-diffusion systems using the semigroup method, it is necessary to understand the $C_0$-semigroup whose generator is the Laplace operator $\Delta$. Therefore, in what follows, we explicitly construct the $C_0$-semigroup generated by $\Delta$ under periodic boundary conditions.

We first introduce several pieces of notation. For all $1 \leq j \leq d$ and $\left( x_1, \dots, x_d \right) \in \mathbb{R}^d$, define
\begin{align*}
  \check{x}_j &= \left( x_1, \dots, x_{j - 1}, 0, x_{j + 1}, \dots, x_d \right),\\
  \hat{x}_j &= \left( x_1, \dots, x_{j - 1}, 1, x_{j + 1}, \dots, x_d \right).
\end{align*}
Let $\Omega = \left[ 0, 1 \right]^d$ and let $C_{\mathrm{per}} \left( \Omega \right)$ be the space of all continuous real-valued periodic functions having period $1$. That is, for all $x \in \Omega$ and $u \in C_{\mathrm{per}} \left( \Omega \right)$, we have $u \left( \check{x}_j \right) = u \left( \hat{x}_j \right)$. Then, $\left( C_{\mathrm{per}} \left( \Omega \right), \left\| \cdot \right\|_{\infty} \right)$ is a Banach space. We define the subset $C^2_{\mathrm{per}} \left( \Omega \right)$ of $C_{\mathrm{per}} \left( \Omega \right)$ by
\begin{align*}
  C^2_{\mathrm{per}} \left( \Omega \right) = \Set{ u \in C_{\mathrm{per}} \left( \Omega \right) | \dfrac{\partial u}{\partial x_j}, \dfrac{\partial^2 u}{\partial x_j^2} \in C_{\mathrm{per}} \left( \Omega \right), 1 \leq j \leq d }.
\end{align*}
Then, we define the operator $\Delta$ on $C^2_{\mathrm{per}} \left( \Omega \right)$ for all $u \in C^2_{\mathrm{per}} \left( \Omega \right)$,
\begin{align*}
  \Delta u = \sum_{j = 1}^d \dfrac{\partial^2 u}{\partial x_j^2}.
\end{align*}
For all $\lambda > 0$ and $t > 0$, define
\begin{align*}
  g_{\lambda, t} \left( x \right) &= \dfrac{1}{\left( 4 \pi \lambda t \right)^{d / 2}} \exp \left( - \dfrac{ \left\| x \right\|^2 }{ 4 \lambda t } \right),\\
  K_{\lambda, t} \left( x \right) &= \sum_{k \in \mathbb{Z}^d} g_{\lambda ,t} \left( x + k \right).
\end{align*}
Then, for $t \geq 0$, define for any $f \in C_{\mathrm{per}} \left( \Omega \right)$ and $x \in \Omega$
\begin{align}\label{eq:rep-heat-semigroup}
  \left( H_{\lambda} \left( t \right) f \right) \left( x \right) =
  \begin{dcases}
    \left( K_{\lambda, t} * f \right) \left( x \right) = \int_{\Omega} K_{\lambda, t} \left( x - y \right) f \left( y \right) dy, & t > 0, \\
    f \left( x \right), & t = 0.
  \end{dcases}
\end{align}

Under this setup, we shall show that $\mathcal{H}_\lambda = \Set{ H_\lambda \left( t \right) }_{t \geq 0}$ is a $C_0$-semigroup and that the generator of $\mathcal{H}_\lambda$ coincides with $\lambda \Delta$ on $C^2_{\mathrm{per}} \left( \Omega \right)$.
\begin{prop}
  The family $\mathcal{H}_\lambda$ is a $C_0$-semigroup on $C_{\mathrm{per}} \left( \Omega \right)$. Moreover, if $A$ is the infinitesimal generator of $\mathcal{H}_\lambda$, the for all $u \in C^2_{\mathrm{per}} \left( \Omega \right)$, 
  \begin{align*}
    A u = \lambda \Delta u.
  \end{align*}
\end{prop}

\begin{proof}
  まず,任意の$t \geq 0$に対して,$H_{\lambda} \left( t \right)$が$C_{\mathrm{per}} \left( \Omega \right)$上の有界線形作用素であることを示す。$t = 0$のときを考える。$H_{\lambda} \left( 0 \right)$は恒等作用素であるから,$C_{\mathrm{per}} \left( \Omega \right)$上の有界線形作用素である。$t > 0$のときを考える。まず,積分の線形性から$H_{\lambda} \left( t \right)$は線形作用素である。次に,任意の$1 \leq j \leq d$に対して,$e_j \in \mathbb{R}^d$を第$j$成分だけが$1$であり,他の成分が$0$であるベクトルとすると,任意の$x \in \Omega$に対して,$\check{x}_j + e_j = \hat{x}_j$が成り立ち,任意の$k \in \mathbb{Z}^d$に対して,$k - e_j \in \mathbb{Z}^d$である。したがって,任意の$u \in C_{\mathrm{per}} \left( \Omega \right)$と$x \in \Omega$に対して,
  \begin{align*}
    \left( H_{\lambda} \left( t \right) u \right) \left( \check{x}_j \right)
    &= \int_{\Omega} \sum_{k \in \mathbb{Z}^d} \dfrac{1}{\left( 4 \pi \lambda t \right)^{d / 2}} \exp \left( - \dfrac{ \left\| \check{x}_j + k \right\|^2 }{ 4 \lambda t } \right) f \left( y \right) dy \\
    &= \int_{\Omega} \sum_{k \in \mathbb{Z}^d} \dfrac{1}{\left( 4 \pi \lambda t \right)^{d / 2}} \exp \left( - \dfrac{ \left\| \hat{x}_j + \left( k - e_j \right) \right\|^2 }{ 4 \lambda t } \right) f \left( y \right) dy \\
    &= \int_{\Omega} \sum_{k \in \mathbb{Z}^d} \dfrac{1}{\left( 4 \pi \lambda t \right)^{d / 2}} \exp \left( - \dfrac{ \left\| \hat{x}_j + k \right\|^2 }{ 4 \lambda t } \right) u \left( y \right) dy \\
    &= \left( H_{\lambda} \left( t \right) u \right) \left( \hat{x}_j \right)
  \end{align*}
  となるため,$H_{\lambda} \left( t \right) u \in C_{\mathrm{per}} \left( \Omega \right)$である。さらに,任意の$x \in \Omega$に対して,$x_l \in \Omega$を$x$に収束する点列とすると,任意の$u \in C_{\mathrm{per}} \left( \Omega \right)$に対して,
  \begin{align*}
    \lim_{l \to \infty} \left( H_{\lambda} \left( t \right) u \right) \left( x_l \right) 
    &= \lim_{l \to \infty} \int_{\Omega} K_{\lambda, t} \left( x_l - y \right) u \left( y \right) dy\\
    &= \int_{\Omega} \lim_{l \to \infty} K_{\lambda, t} \left( x_l - y \right) u \left( y \right) dy\\
    &= \int_{\Omega} K_{\lambda, t} \left( x - y \right) u \left( y \right) dy\\
    &= \left( H_{\lambda} \left( t \right) u \right) \left( x \right) 
  \end{align*}
  となる。よって,$H_{\lambda} \left( t \right)$は$C_{\mathrm{per}} \left( \Omega \right)$上の作用素である。最後に,任意の$u \in C_{\mathrm{per}} \left( \Omega \right)$に対して,
  \begin{align*}
    \left\| H_{\lambda} \left( t \right) u \right\|_\infty
    &= \left\| K_{\lambda, t} * u \right\|_\infty\\
    &\leq \left\| K_{\lambda, t} \right\|_{L^1} \left\| u \right\|_\infty\\
    &= \left\| u \right\|_\infty
  \end{align*}
  となるため,$H_{\lambda} \left( t \right)$は有界である。以上のことから,$H_{\lambda} \left( t \right)$は$C_{\mathrm{per}} \left( \Omega \right)$上の有界線形作用素である。
 
  次に,任意の$t \geq 0$に対して,$\mathcal{H}_{\lambda} \left( t \right)$が$C_0$-semigroupであることを示す。まず,$H_{\lambda} \left( 0 \right)$は恒等作用素である。次に,任意の$s, t \geq 0$に対して,$s = 0$または$t = 0$ならば$S \left( s + t \right) = S \left( s \right)S \left( t \right)$は明らかである。$s ,t > 0$のとき,任意の$u \in C_{\mathrm{per}} \left( \Omega \right)$と$x \in \Omega$に対して,
  \begin{align*}
    \left( H_{\lambda} \left( s \right) H_{\lambda} \left( t \right) u \right) \left( x \right)
    &= \int_{\Omega} K_{\lambda, s} \left( x - z \right) \int_{\Omega} K_{\lambda, t} \left( x - y \right) u \left( y \right) dy dz\\
    &= \int_{\Omega} \left[ \left( K_{\lambda, s} * K_{\lambda, t} \right) \left( x - y \right) \right] u \left( y \right) dy\\
    &= \int_{\Omega} K_{\lambda, s + t} \left( x - y \right) u \left( y \right) dy\\
    &= \left( H_{\lambda} \left( s + t \right) u \right) \left( x \right)
  \end{align*}
  となる。よって,$\mathcal{H}_{\lambda} \left( t \right)$はsemigroupである。さらに,任意の$x \in \Omega$と$\delta > 0$に対して,$B_\delta\left( x \right) = \set{ y \in \Omega | \left\| x - y \right\|_2 < \delta }$とおけば,任意の$u \in C_{\mathrm{per}} \left( \Omega \right)$に対して,
  \begin{align*}
    \left| \left( H_{\lambda} \left( t \right) u \right) \left( x \right)  - u \left( x \right) \right|
    &= \left| \int_{\Omega} K_{\lambda, t} \left( x - y \right) u \left( y \right) dy - u \left( x \right) \right| \\
    &= \left| \int_{\Omega} K_{\lambda, t} \left( x - y \right) \left( u \left( y \right) - u \left( x \right) \right) dy \right| \\
    &\leq \int_{B_\delta \left( x \right) \cap \Omega} K_{\lambda, t} \left( x - y \right) \left| u \left( y \right) - u \left( x \right) \right| dy + \int_{\left[B_\delta \left( x \right)\right]^c \cap \Omega} K_{\lambda, t} \left( x - y \right) \left| u \left( y \right) - u \left( x \right) \right| dy,
  \end{align*}
  となるが,第1項は$u$の$\Omega$上での一様収束性により$\delta \to 0$において$0$に収束して,第2項は$t \to 0$における$g_{\lambda, t}$の集中により$\delta \to 0$において$0$に収束する。$\mathcal{H}_{\lambda} \left( t \right)$は$C_0$-semigroupである。

  任意の$t > 0$と$u \in C^2_{\mathrm{per}} \left( \Omega \right)$に対して,
  \begin{align*}
    \left( \dfrac{d}{dt} H_{\lambda} \left( t \right) u \right) \left( x \right)
    &= \int_\Omega \dfrac{\partial}{\partial t} K_{\lambda, t} \left( x - y \right) u \left( y \right) dy\\
    &= \lambda \int_\Omega K_{\lambda, t} \left( x - y \right) u \left( y \right) dy \\
    &= \lambda \Delta \int_\Omega \Delta K_{\lambda, t} \left( x - y \right) u \left( y \right) dy\\
    &= \lambda \Delta \left( H_{\lambda} \left( t \right) u \right) \left( x \right)\\
    &= \left( H_{\lambda} \left( t \right) \lambda \Delta u \right) \left( x \right)
  \end{align*}
  となる。よって,両辺を$0$から$t$まで積分することにより,
  \begin{align*}
    \left( H_\lambda \left( t \right) u \right) \left( x \right) - u \left( x \right) = \int_0^t \left( H_{\lambda} \left( s \right) \lambda \Delta u \right) \left( x \right) ds
  \end{align*}
  となる。したがって,
  \begin{align*}
    \left\| \dfrac{H_\lambda \left( t \right) u - u \left( x \right)}{t} - \lambda \Delta u \right\|_\infty
    &\leq \dfrac{1}{t} \int_0^t \left\| H_{\lambda} \left( s \right) \lambda \Delta u - \lambda \Delta u \right\|_\infty ds \to 0 ~ \left( t \searrow 0 \right)
  \end{align*}
  となるので,
  \begin{align*}
    \lim_{t \searrow 0} \dfrac{H_\lambda \left( t \right) u - u \left( x \right)}{t} = \lambda \Delta u
  \end{align*}
  を得る。
\end{proof}

\subsubsection{Application to reaction-diffusion systems}

Up to this point, we have explicitly constructed a $C_0$-semigroup on $C_{\mathrm{per}} \left( \Omega \right)$ whose infinitesimal generator coincides with $\Delta$ on $C^2_{\mathrm{per}} \left( \Omega \right)$. 
We now turn to the application of this framework to reaction-diffusion systems. Since a reaction-diffusion system consists of several coupled partial differential equations, it is natural to work on a product of Banach spaces and to consider $C_0$-semigroups acting componentwise.

Let $n$ be a positive integer. For any $U = \left( u_1, \cdots, u_n \right) \in \left( C_{\mathrm{per}} \left( \Omega \right) \right)^n$, define
\begin{align*}
  \left\| U \right\|_{\infty} = \max \set{ \left\| u_1 \right\|_\infty, \dots, \left\| u_n \right\|_\infty }.
\end{align*}
Then $\left( \left( C_{\mathrm{per}} \left( \Omega \right) \right)^n, \left\| \cdot \right\|_\infty \right)$ is a Banach space. 
For $1 \leq i \leq n$, let $\mathcal{S}_i = \Set{ S_i \left( t \right) }_{t \ge 0}$ be a $C_0$-semigroup on $C_{\mathrm{per}} \left( \Omega \right)$, and define $\mathcal{S} = \Set{ S \left( t \right) }_{t \ge 0}$ by
\begin{align*}
  S \left( t \right) \left( U \right) = \left( S_1 \left( t \right) u_1, \dots, S_n \left( t \right) u_n \right).
\end{align*}
The following two propositions show that this family $\mathcal{S}$ is a $C_0$-semigroup on the product space and describe its infinitesimal generator.

\begin{prop}\label{prop:product-of-semigroup}
  $\mathcal{S}$ is a $C_0$-semigroup on $\left( C_{\mathrm{per}} \left( \Omega \right) \right)^n$.
\end{prop}

\begin{proof}
  Let $U = \left( u_1, \cdots, u_n \right) \in \left( C_{\mathrm{per}} \left( \Omega \right) \right)^n$. First, we can compute
  \begin{align*}
     S \left( 0 \right) \left( U \right)
     &= \left( S_1 \left( 0 \right) u_1, \dots, S_n \left( 0 \right) u_n \right) \\
     &= \left( u_1, \cdots, u_n \right)\\
     &= U,
  \end{align*}
  then $S \left( 0 \right)$ is the identity operator on $\left( C_{\mathrm{per}} \left( \Omega \right) \right)^n$. Next, for any $s, t \geq 0$, 
  \begin{align*}
    S \left( s + t \right) \left( U \right)
     &= \left( S_1 \left( s + t \right) u_1, \dots, S_n \left( s + t \right) u_n \right) \\
     &= \left( S_1 \left( s \right) S_1 \left( t \right) u_1, \cdots, S_n \left( s \right) S_n \left( t \right) u_n \right)\\
     &= S \left( s \right) S \left( t \right) U,
  \end{align*}
  and hence we have $S \left( s + t \right) = S \left( s \right) S \left( t \right)$. Finally,
  \begin{align*}
    \lim_{t \searrow 0} S \left( t \right) \left( U \right)
    &= \left( \lim_{t \searrow 0} S_1 \left( t \right) u_1, \dots, \lim_{t \searrow 0} S_n \left( t \right) u_n \right) \\
    &= \left( u_1, \cdots, u_n \right)\\
    &= U.
  \end{align*}
  Therefore, the family $S \left( t \right) U$ is a $C_0$-semigroup on $\left( C_{\mathrm{per}} \left( \Omega \right) \right)^n$.
\end{proof}

\begin{prop}\label{prop:product-of-generator}
  For each $1 \leq i \leq n$, let $A_i$ and $A$ denote the infinitesimal generator of $\mathcal{S}_i$ and $\mathcal{S}$, respectively. Then, for any
  \begin{align*}
    U = \left( u_1, \cdots, u_n \right) \in D \left( A_1 \right) \times \cdots \times D \left( A_n \right),
  \end{align*}
  we have
  \begin{align*}
    AU = \left( A_1 u_1, \dots, A_n u_n \right).
  \end{align*}
\end{prop}

\begin{proof}
  For any $U = \left( u_1, \cdots, u_n \right) \in D \left( A_1 \right) \times \cdots \times D \left( A_n \right)$, we have
  \begin{align*}
    AU
    &= \lim_{t \searrow 0} \dfrac{ S \left( t \right) U - U }{t} \\
    &= \lim_{t \searrow 0} \left( \dfrac{ S_1 \left( t \right) u_1 - u_1 }{t}, \dots, \dfrac{ S_n \left( t \right) u_n - u_n }{t} \right) \\
    &= \left( A_1 u_1, \dots, A_n u_n \right),
  \end{align*}
  as claimed.
\end{proof}

Let us confirm how the solution of a reaction-diffusion system can be expressed within the framework of semigroup theory. In general, as already stated in \eqref{eq:general-reaction-diffusion-equations}, a reaction–diffusion system is given in the form
\begin{align*}
  \begin{dcases}
    \dfrac{\partial u_i}{\partial t} \left( x, t \right) = \lambda_i \Delta u_i \left( x, t \right) + R_i \left( u_1 \left( x, t \right), \dots, u_n \left( x, t \right) \right), & \left( x, t \right) \in \Omega \times \left( 0, T \right], \\
    u_i \left( x, 0 \right) = \tilde{u}_i \left( x \right), & x \in \Omega\\
    1 \leq i \leq n,\\
  \end{dcases}
\end{align*}
where $\Omega$ is equipped with periodic boundary conditions, and $\tilde{U} = \left( \tilde{u}_1, \dots, \tilde{u}_n \right) \in \left( C^2_{\mathrm{per}} \left( \Omega \right) \right)^n$ as initial value. Also let $D = \left( \lambda_1, \cdots, \lambda_n \right)$. For any $U \left( \cdot, t \right) = \left( u_1 \left( \cdot, t \right), \dots, u_n \left( \cdot, t \right) \right) \in \left( C^2_{\mathrm{per}} \left( \Omega \right) \right)^n$, we define
\begin{align*}
  \Delta_{D} U \left( \cdot, t \right) = \left( \lambda_1 \Delta u_1 \left( \cdot, t \right), \dots, \lambda_n \Delta u_n \left( \cdot, t \right) \right),
\end{align*}
and for all $0 \leq t \leq T$, 
\begin{align*}
  R \left( U \left( \cdot, t \right) \right) = \left(R_1 \left( u_1 \left( \cdot, t \right), \dots, u_n \left( \cdot, t \right) \right), \dots, R_n \left( u_1 \left( \cdot, t \right), \dots, u_n \left( \cdot, t \right) \right) \right).
\end{align*}
With this notation, the reaction-diffusion system \eqref{eq:general-reaction-diffusion-equations} can be written in the compact form
\begin{align*}
  \begin{dcases}
    \dfrac{ \partial U }{ \partial t } \left( x, t \right) = \Delta_{D} U \left( x, t \right) + R \left( U \left( x, t \right) \right), & \left( x, t \right) \in \Omega \times \left( 0, T \right] \\
    U \left( x, 0 \right) = \tilde{U} \left( x \right), & x \in \Omega
  \end{dcases}
\end{align*}
From Proposition \ref{prop:product-of-semigroup} we know $\Delta_D$ generates a $C_0$-semigroup. Let this $C_0$-semigroup be denoted by $\mathcal{H}_D = \Set{ H_D \left( t \right) }_{t \geq 0}$. Then Proposition \ref{prop:product-of-generator} ensures that for any $U = \left( u_1, \cdots, u_n \right) \in \left( C^2_{\mathrm{per}} \left( \Omega \right) \right)^n$,
\begin{align*}
  H_{D} \left( t \right) U = \left( H_{\lambda_1} \left( t \right) u_1, \dots, H_{\lambda_n} \left( t \right) u_n \right)
\end{align*}
Assuming that $R$ is Lipschitz continuous, the classical solution of the reaction-diffusion system \eqref{eq:general-reaction-diffusion-equations} is represented by
\begin{align*}
  U \left( \cdot, t \right) = H_{D} \left( t \right) \tilde{U} \left( \cdot \right) + \int_0^t H_D \left( t - s \right) R \left( U \left( \cdot, s \right) \right) ds.
\end{align*}
Consequently, for each $1 \leq i \leq n$, we have
\begin{align*}
  u_i \left( \cdot, t \right) = H_{\lambda_i} \left( t \right) \tilde{u}_i \left( \cdot \right) + \int_0^t H_{\lambda_i} \left( t - s \right) R_i \left( u_1 \left( \cdot, s \right), \dots, u_n \left( \cdot, s \right) \right) ds.
\end{align*}
