
\begin{figure}[h]
  \centering
  \includegraphics[width=0.95\linewidth]{figures/view_u_tilde_series.png}
  % \caption{本図は,Gray--Scottモデルにおける$u_{1, \theta}$ のパターンと,それに対応するパーシステンス図の時間発展を比較したものである。横方向は時刻$t = 0,375,750,\dots, 3000$ を表し,各列の上段に$u_{1, \theta}$のスナップショット,下段にその時刻でのパーシステンス図$\mathrm{dgm}_p~(u_{1, \theta})$を示している。ただし,パーシステンス図中の青色の点および橙色の点はそれぞれ$0$次のホモロジーおよび$1$次のホモロジーに対応している。各列には,上から順に$\theta_1$,$\theta_1^\prime$,$\theta_2$,$\theta_2^\prime$の結果が並んでいる。パーシステン鈴に対して,各ペア$\left( \theta_1,\theta_1^\prime \right)$および$\left( \theta_2, \theta_2^\prime \right)$の間で点配置は類似しており,パラメータの微小な差がパーシステンス図に与える影響が小さいことが視覚的に確認できる。}
  \caption{This figure compares the temporal evolution of patterns of $u_{1,\theta}$ in the Gray--Scott model and the corresponding persistence diagrams. The horizontal axis indicates the time points $t = 0, 375, 750, \dots, 3000$. In each row, the upper panel shows a snapshot of $u_{1,\theta}$, and the lower panel shows the persistence diagram $\mathrm{dgm}_p~\left( u_{1,\theta} \right)$ at the same time. In the persistence diagrams, blue and orange points correspond to $0$-th and $1$-th homology, respectively. The results for $\theta_1$, $\theta_1^\prime$, $\theta_2$, and $\theta_2^\prime$ are arranged from top to bottom. For each pair $\left( \theta_1, \theta_1^\prime \right)$ and $\left( \theta_2, \theta_2^\prime \right)$, the point configurations in the persistence diagrams are similar, visually confirming that small differences in the parameters have only a small effect on the persistence diagrams.}
  \label{fig:theta-time-series}
\end{figure}
