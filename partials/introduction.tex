\section{Introduction}

% According to Alan Turing's work, characteristic surface patterns observed in animals and plants can spontaneously emerge through the interaction and diffusion of hypothetical substances called morphogens \cite{turing1990chemical}. This phenomenon is modeled by a system of reaction-diffusion equations, and the spatial patterns generated by this phenomenon are called Turing patterns. For instance, it has been reported that a variety of animal and plant surface patterns can be explained as Turing patterns \cite{kondo1995reaction, gravan2004evolving, liu2006two, de2020leopard}, and Turing patterns are regarded as an important mathematical model for studying biological morphogenesis.

% この現象は反応拡散方程式系によってモデル化され,この現象によって生成される空間パターンはTuringパターンと呼ばれる。

% This phenomena are modeled by reaction-diffusion systems, and stationary spatial patterns arising from diffusion-driven instability are called Turing patterns. 

According to Alan Turing's seminal work \cite{turing1990chemical}, characteristic patterns on the body surfaces of animals and plants can arise spontaneously from the coupled processes of reaction and diffusion of chemical substances, often referred to as morphogens. Such pattern formation is commonly modeled using reaction-diffusion systems, and the resulting spatial patterns are often referred to as Turing patterns. A variety of animal and plant surface patterns have been modeled within this framework \cite{kondo1995reaction, gravan2004evolving, liu2006two, de2020leopard}, making Turing patterns an important mathematical model for studying biological morphogenesis.

While evaluation of the similarity of Turing patterns has often relied on visual inspection, quantitative evaluation on the basis of topological data analysis (TDA) has attracted increasing attention in recent years \cite{volkening2024methods}. TDA is a framework for quantitatively classifying the shapes of data and has advanced substantially recently \cite{carlsson2009topology}. In particular, persistent homology \cite{edelsbrunner2002topological} is one of the representative methods in TDA: by computing homology over a sequence of spaces obtained from data, it describes the ``shape'' of the data. TDA, including persistent homology, is applied not only to the analysis of Turing patterns but also across a wide range of fields, including materials science \cite{nakamura2015persistent, hiraoka2016hierarchical}, protein science \cite{gameiro2015topological}, and image analysis \cite{tanabe2021homological}.

Applications of persistent homology to pattern-forming systems, including Turing patterns, are already wide-ranging, such as the analysis of coral resilience models \cite{mcdonald2023zigzag}, zebrafish stripe patterns \cite{mcguirl2020topological}, quantitative comparisons of fish skin patterns \cite{kanamori2025analysis}, and the development of clustering algorithms for Turing patterns \cite{spector2026persistent}. In particular, \cite{spector2026persistent} enables the parameter space to be partitioned into clusters of parameters for reaction-diffusion systems that generate similar patterns. This result indicates that there is a close relationship between the parameters of reaction-diffusion systems and the persistent homology obtained from the Turing patterns generated by the systems with those parameters.

The results of persistent homology are usually summarized by a set of persistence diagrams, and comparisons between persistence diagrams are carried out using a distance known as the bottleneck distance. In this context, the aim of this study is to evaluate how the persistence diagrams derived from Turing patterns vary under changes in the parameters of reaction-diffusion systems using the bottleneck distance. More specifically, we prove that, under certain conditions on the reaction-diffusion system, persistence diagrams are Lipschitz continuous with respect to the model parameters. We also show, as a concrete example, that the Gray--Scott model \cite{gray1983autocatalytic, gray1984autocatalytic}, a representative reaction-diffusion system, can satisfy these conditions.

% パーシステントホモロジーの解析結果は,通常,パーシステンス図という集合に要約され,パーシステンス図の定量比較は,ボトルネック距離と呼ばれる距離によって行われる。そこで,本研究では,拡散反応系のパラメータの変動によって,得られるパーシステンス図がどのように変動するかをボトルネック距離によって評価し,拡散反応系のパラメータとパーシステントホモロジーの関係性を解析する。より具体的には,拡散反応系がある条件を満たすときに,パーシステンス図が拡散反応系のパラメータに対してリプシッツ連続になることを証明する。また,代表的な拡散反応系であるGray--Scott model \cite{gray1983autocatalytic, gray1984autocatalytic}がこの条件を満たしうることを具体例として証明する。

% This study aim to we analyze the relationship between the parameters of reaction-diffusion systems and the persistence diagrams obtained from the Turing patterns.

% The objective of this study is to analyze the relationship between the parameters of reaction-diffusion systems and persistent homology by investigating how the persistence diagram, a summary of persistent homology, varies under perturbations of the parameters in reaction-diffusion systems.
% It is known that the set of persistence diagrams forms a metric space when equipped with a distance called the bottleneck distance. Accordingly, in this study we prove that, under certain conditions, persistence diagrams are Lipschitz continuous with respect to the parameters of a reaction-diffusion system.
% The output of persistent homology is usually summarized and analyzed via persistence diagrams. In this context, we analyze the relationship between the parameters of reaction-diffusion systems and the persistence diagrams obtained from the Turing patterns. Specifically, under certain conditions, we show that the persistence diagrams obtained from Turing patterns are Lipschitz continuous with respect to the model parameters of the reaction-diffusion system. We also prove that this Lipschitz continuity naturally holds for the Gray--Scott model \cite{gray1983autocatalytic, gray1984autocatalytic}, which is a representative reaction-diffusion system.

To achieve the objectives of this paper, Section \ref{sec:preliminary} reviews the basic concepts of persistent homology, reaction-diffusion systems, and the semigroup method needed for this study. Section \ref{sec:method} defines the general reaction-diffusion system considered in this paper and discusses the relationship between its model parameters and persistent homology. Section \ref{sec:example} applies the discussion in Section \ref{sec:method} to the Gray--Scott model, a representative reaction-diffusion system. Section \ref{sec:simulation-study} visualizes the results obtained in Sections \ref{sec:method} and \ref{sec:example} through numerical experiments. Finally, Section \ref{sec:conclusion} summarizes the conclusions of this paper and discusses the limitations of the results and directions for future work.


% Alan Turingの研究によれば,動植物に見られる特徴的な体表パターンは,morphogenと呼ばれる仮想的な物質の反応と拡散の相互作用によって自発的に形成されうる。この相互作用は反応拡散方程式系としてモデル化され,拡散誘起不安定性により生じる定常的な空間パターンはTuringパターンと呼ばれる。実際にさまざまな動植物の体表パターンがTuringパターンとして説明可能であると報告されており,Turingパターンは生物の形態形成を調べる上で重要な数理モデルと考えられている。

% Turingパターンの類似性は視覚的に評価されることが多かったが,近年では,トポロジカルデータ解析を活用した類似性の定量的評価が注目を集めている。トポロジカルデータ解析は,データの形状を定量的に分類する手法であり,近年大きな発展を遂げている。特に,パーシステントホモロジーはトポロジカルデータ解析の代表的な手法であり,データから得られる空間列に対してホモロジーを計算することによって,データの「形」を記述する。パーシステントホモロジーを始めとするTDAは,Turingパターンの解析に限らず,材料科学,タンパク質科学,画像解析などのさまざまな分野に応用されている。

% Turingパターンを含めたパターン形成系へのパーシステントホモロジーの応用もすでに多岐にわたっており,サンゴのレジリエンスモデル解析 \cite{mcdonald2023zigzag},ゼブラフィッシュの縞模様解析 \cite{mcguirl2020topological},魚類の皮膚パターンの定量比較 \cite{kanamori2025analysis},Turingパターンのクラスタリングアルゴリズムの開発 \cite{spector2026persistent}などがある。特に,論文\cite{spector2026persistent}は,類似したパターンを生成する反応拡散方程式のパラメータのクラスタにパラメータ空間を分割することを可能にしている。この結果は,反応拡散方程式のパラメータとそのパラメータが生成するTuringパターンから得られるパーシステンスホモロジーの間に密接な関係があることを示している。

% パーシステントホモロジーの解析結果は,パーシステンス図と呼ばれる集合に要約して扱われる。そこで,本研究では,反応拡散方程式のパラメータとそのパラメータが生成するTuringパターンから得られるパーシステンス図の関係を解析する。具体的には,特定の条件のもとで,Turingパターンから得られるパーシステンス図が反応拡散方程式のモデルパラメータに対してLipshitz連続であることを示す。ここで,パーシステンス図とは,パーシステントホモロジーを要約した集合である。また,Lipshitz連続性が,代表的な反応拡散方程式であるGray--Scottモデルで自然に成立することを証明する。

% 本論文の目的を達成するために,まず Section \ref{sec:preliminary} では,本研究に必要なパーシステントホモロジー,反応拡散方程式系,および半群法の基礎事項を概説する。Section \ref{sec:method}では,本論文で扱う一般的な反応拡散方程式系を定義し,そのモデルパラメータとパーシステントホモロジーとの関係を議論する。Section \ref{sec:example}では,Section \ref{sec:method}の議論を代表的な反応拡散方程式系である Gray--Scott モデルに適用する。Section \ref{fig:simulation-for-inequality}では,Section \ref{sec:method}および Section \ref{sec:example}で得られた結果を数値実験により可視化する。最後に Section \ref{sec:conclusion}では,本論文の結論をまとめ,得られた結果の限界と今後の課題を議論する。
