\section{Introduction}

According to Alan Turing's study, characteristic surface patterns observed in animals and plants can spontaneously emerge through the interplay of reaction and diffusion of hypothetical substances called morphogens \cite{turing1990chemical}. This interaction is modeled by reaction--diffusion systems, and stationary spatial patterns arising from diffusion-driven instability are called Turing patterns. For instance, it has been reported that a variety of animal and plant surface patterns can be explained as Turing patterns \cite{kondo1995reaction, gravan2004evolving, liu2006two, de2020leopard}, and Turing patterns are regarded as an important mathematical model for studying biological morphogenesis.

While the similarity of Turing patterns has often been evaluated visually, quantitative evaluation based on topological data analysis (TDA) has attracted increasing attention in recent years \cite{volkening2024methods}. TDA is a framework for quantitatively classifying the shapes of data and has advanced substantially over the past decade \cite{carlsson2009topology}. In particular, persistent homology \cite{edelsbrunner2002topological} is one of the representative methods in TDA: by computing homology over a sequence of spaces obtained from data, it describes the ``shape'' of the data. Moreover, by defining a distance between sets called persistence diagrams derived from persistent homology, persistence diagrams can be compared. Through comparisons of persistence diagrams obtained from Turing patterns, it becomes possible to quantitatively assess Turing patterns.

Applications of persistent homology to pattern-forming systems, including Turing patterns, are already wide-ranging, such as the analysis of coral resilience models \cite{mcdonald2023zigzag}, zebrafish stripe patterns \cite{mcguirl2020topological}, quantitative comparisons of fish skin patterns \cite{kanamori2025analysis}, and the development of clustering algorithms for Turing patterns \cite{spector2026persistent}. In particular, \cite{spector2026persistent} enables the parameter space to be partitioned into clusters of parameters for reaction-diffusion systems that generate similar patterns. This result indicates that there is a close relationship between the parameters of reaction-diffusion systems and the persistence diagrams obtained from the Turing patterns generated by those parameters.

In this study, we analyze the relationship between the parameters of reaction-diffusion systems and the persistence diagrams obtained from the Turing patterns generated by those parameters. Specifically, under certain conditions, we show that the persistence diagrams obtained from Turing patterns are Lipschitz continuous with respect to the model parameters of the reaction-diffusion system. We also prove that this Lipschitz continuity naturally holds for the Gray--Scott model \cite{gray1983autocatalytic, gray1984autocatalytic}, which is a representative reaction-diffusion system.

To achieve the objectives of this paper, Section \ref{sec:preliminary} reviews the basic concepts of persistent homology, reaction-diffusion systems, and the semigroup method needed for this study. Section \ref{sec:method} defines the general reaction-diffusion system considered in this paper and discusses the relationship between its model parameters and persistent homology. Section \ref{sec:example} applies the discussion in Section \ref{sec:method} to the Gray--Scott model, a representative reaction-diffusion system. Section \ref{sec:simulation-study} visualizes the results obtained in Sections \ref{sec:method} and \ref{sec:example} through numerical experiments. Finally, Section \ref{sec:conclusion} summarizes the conclusions of this paper and discusses the limitations of the results and directions for future work.


% Alan Turingの研究によれば,動植物に見られる特徴的な体表パターンは,morphogenと呼ばれる仮想的な物質の反応と拡散の相互作用によって自発的に形成されうる。この相互作用は反応拡散方程式系としてモデル化され,拡散誘起不安定性により生じる定常的な空間パターンはTuringパターンと呼ばれる。実際にさまざまな動植物の体表パターンがTuringパターンとして説明可能であると報告されており,Turingパターンは生物の形態形成を調べる上で重要な数理モデルと考えられている。

% Turingパターンの類似性は視覚的に評価されることが多かったが,近年では,トポロジカルデータ解析を活用した類似性の定量的評価が注目を集めている。トポロジカルデータ解析は,データの形状を定量的に分類する手法であり,近年大きな発展を遂げている。特に,パーシステントホモロジーはトポロジカルデータ解析の代表的な手法であり,データから得られる空間列に対してホモロジーを計算することによって,データの「形」を記述する。このパーシステントホモロジーから得られるパーシステンス図と呼ばれる集合の間に距離を定めることによって,パーシステンス図の比較が可能になる。さらに,Turingパターンから得られるパーシステンス図の比較を通して,Turingパターンの定量的評価も可能になる。

% Turingパターンを含めたパターン形成系へのパーシステントホモロジーの応用はすでに多岐にわたっており,サンゴのレジリエンスモデル解析 \cite{mcdonald2023zigzag},ゼブラフィッシュの縞模様解析 \cite{mcguirl2020topological},魚類の皮膚パターンの定量比較 \cite{kanamori2025analysis},Turingパターンのクラスタリングアルゴリズムの開発 \cite{spector2026persistent}などがある。特に,論文\cite{spector2026persistent}は,類似したパターンを生成する反応拡散方程式のパラメータのクラスタにパラメータ空間を分割することを可能にしている。この結果は,反応拡散方程式のパラメータとそのパラメータが生成するTuringパターンから得られるパーシステンス図に密接な関係があることを示している。

% 本研究では,反応拡散方程式のパラメータとそのパラメータが生成するTuringパターンから得られるパーシステンス図の関係を解析する。具体的には,特定の条件のもとで,Turingパターンから得られるパーシステンス図が反応拡散方程式のモデルパラメータに対してLipshitz連続であることを示す。また,Lipshitz連続性が,代表的な反応拡散方程式であるGray--Scottモデルで自然に成立することを証明する。

% 本論文の目的を達成するために,まず Section \ref{sec:preliminary} では,本研究に必要なパーシステントホモロジー,反応拡散方程式系,および半群法の基礎事項を概説する。Section \ref{sec:method}では,本論文で扱う一般的な反応拡散方程式系を定義し,そのモデルパラメータとパーシステントホモロジーとの関係を議論する。Section \ref{sec:example}では,Section \ref{sec:method}の議論を代表的な反応拡散方程式系である Gray--Scott モデルに適用する。Section \ref{fig:simulation-for-inequality}では,Section \ref{sec:method}および Section \ref{sec:example}で得られた結果を数値実験により可視化する。最後に Section \ref{sec:conclusion}では,本論文の結論をまとめ,得られた結果の限界と今後の課題を議論する。
