\section{Introduction}

% Alan Turingの研究によれば,生物の特徴的な体表パターンは,モルフォゲンと呼ばれる物質が生体内で相互作用することによって生じる \cite{turing1990chemical}。この相互作用は反応拡散方程式系という偏微分方程式系でモデル化される。反応拡散方程式系によって生成される空間パターンはTuringパターンを呼ばれ,実際に多くの生物の体表パターンがTuringパターンであることが示されている\cite{kondo1995reaction, gravan2004evolving, liu2006two, ishida2019constructing}。

% 生物の体表パターンを分類する上で,Turingパターンの類似性の評価は非常に重要であるが,この類似性は視覚的に評価されることが多かった。しかし,
% 近年ではパーシステントホモロジーを活用した定量的評価が注目を集めている\cite{volkening2024methods}。パーシステントホモロジー\cite{edelsbrunner2002topological}は,トポロジーの方法論を応用して,データの「かたち」を定量的に記述するための手法であり,近年急速な発展を遂げている。

% パーシステントホモロジーを利用したTuringパターンの解析については,これまで多くの研究が行われている[]。



% パターン形成系に対するパーシステントホモロジーの応用はすでに多岐にわたっている。Kanamoriらの研究はパーシステントホモロジーがTuringパターンの定量的な類似性評価に有用であることを示している。また,

% Spectorらの研究は,パーシステントホモロジーを活用してTuringパターンをクラスタリングするアルゴリズムを開発している。

% これらの研究は,十分に類似したTuringパターンからは十分に類似したパーシステントホモロジーが得られることを示唆している。一方で,パーシステントホモロジーを活用したTuringパターンの解析研究は,どれも数値実験的な研究にとどまっている。このような背景を踏まえて,本論文の目的は,Turingパターンを生成する反応拡散方程式系の類似性とパーシステントホモロジーの類似性の間に成り立つ関係性を理論的に検討することである。具体的には,Turingパターンから得られるパーシステンス図の間のボトルネック距離が,Turingパターンを生成する反応拡散方程式系のモデルパラメータに対してLipschitz連続であるための条件を検討する。

% 本論文の目的を達成するために,まずSection \ref{sec:preliminary}で必要最低限のパーシステントホモロジーおよび反応拡散方程式系の知識を整理する。Section \ref{sec:method}では,実際に検討する一般的な反応拡散方程式系を定義し,反応拡散方程式系のモデルパラメータとパーシステントホモロジーの間の関係性を議論する。Section \ref{sec:example}では,Section \ref{sec:method}での議論を代表的な反応拡散方程式系であるGray--Scottモデルに適用する。Section \ref{sec:simulation}では,Section \ref{sec:example}で得られた結果を数値実験を利用して視覚化する。最後に,Section \ref{sec:conclusion}では,本論文で得られた結果とその限界,および,今後の研究課題を議論する。
