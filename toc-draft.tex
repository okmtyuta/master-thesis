\documentclass{thesis}

\addbibresource{references.bib}

\begin{document}

\begin{titlepage}
  \begin{center}
    \vspace*{12truemm}

    {\LARGE
      修士論文\\
    }

    \vspace*{12truemm}

    {\LARGE
      Lipschitz Continuity of Persistence Diagrams Derived from Turing Patterns \\
      チューリングパターンから得られるパーシステンス図のリプシッツ連続性 \\
    }
  \end{center}

  \vspace*{12truemm}

  \begin{center}
    {\LARGE
      指導教員 田中 利幸 教授\\
    }
  \end{center}

  \vspace*{12truemm}

  \begin{center}
    {\LARGE
      京都大学大学院情報学研究科\\
      情報学専攻データ科学コース\\
      修士課程\\
      令和6年度入学\\
    }
    \vspace*{12truemm}
    {\LARGE
      氏名 岡本優太\\
    }
    \vspace*{12truemm}
    {\LARGE
      令和8年1月提出
    }
    \vspace*{15truemm}
  \end{center}
\end{titlepage}

\tableofcontents

\section{Introduction}

% Turingパターンの概要を端的に述べて,近年パーシステントホモロジーを活用したTuringパターンの類似性評価が注目を受けていることを述べる。

% パーシステントホモロジーの概要を述べる。パーシステントホモロジーが

% 生物の複雑な体表パターンは,反応拡散系と呼ばれる偏微分方程式系によって生成されるTuringパターンによって説明できると考えられている。Turingパターン間の類似性は,主に視覚的に評価されてきたが,近年ではパーシステントホモロジーを利用したTuringパターンの定量的な評価が注目されている。

% パーシステントホモロジーは,ホモロジーという数学の理論を活用して,データの「形状」を調べる技術であり,広い分野で応用されている。

% Turingパターンの概要から述べ始めて,Turingパターンをパーシステントホモロジーで解析する方法が近年注目されていることを述べる。Turingパターンをパーシステントホモロジーで解析している研究をいくつか述べつつ,いくつかの研究からTuringパターンを生成する反応拡散系の微小な変化に対して,パーシステントホモロジーも微小にしか変化しないことが示唆されていることを述べる。本研究の目的として,既存研究の数値実験の結果を理論的に表現することであると述べる。

Alan Turingの研究によれば,動植物の特徴的な体表パターンは,モルフォゲンと呼ばれる物質が生体内で相互作用することによって生じる。この相互作用は反応拡散方程式系という偏微分方程式系でモデル化される。反応拡散方程式系によって生成される空間パターンはTuringパターンを呼ばれ,実際に多くの動植物の体表パターンがTuringパターンであることが示されている。

Turingパターンの類似性は視覚的に評価されることが多いが,近年ではトポロジカルデータ解析を活用したTuringパターンの類似性の定量的評価が注目を集めている。トポロジカルデータ解析は,データの形状を定量的に分類するための手法であり,近年大きな注目を集めている。トポロジカルデータ解析の代表的手法であるパーシステントホモロジーは,時系列的に変化させたデータのホモロジーを計算することで,データの「形」を記述する。パーシステントホモロジーはパーシステンス図と呼ばれる多重集合を利用して要約され,パーシステンス図の類似性はボトルネック距離と呼ばれる距離関数によって定量的に評価される。

パターン形成系に対するパーシステントホモロジーの応用はすでに多岐にわたっている。Kanamoriらの研究はパーシステントホモロジーがTuringパターンの定量的な類似性評価に有用であることを示している。また,Spectorらの研究は,パーシステントホモロジーを活用してTuringパターンをクラスタリングするアルゴリズムを開発している。

これらの研究は,十分に類似したTuringパターンからは十分に類似したパーシステントホモロジーが得られることを示唆している。一方で,パーシステントホモロジーを活用したTuringパターンの解析研究は,どれも数値実験的な研究にとどまっている。このような背景を踏まえて,本論文の目的は,Turingパターンを生成する反応拡散方程式系の類似性とパーシステントホモロジーの類似性の間に成り立つ関係性を理論的に検討することである。具体的には,Turingパターンから得られるパーシステンス図の間のボトルネック距離が,Turingパターンを生成する反応拡散方程式系のモデルパラメータに対してLipschitz連続であるための条件を検討する。

本論文の目的を達成するために,まずSection \ref{sec:preliminary}で必要最低限のパーシステントホモロジーおよび反応拡散方程式系の知識を整理する。Section \ref{sec:method}では,実際に検討する一般的な反応拡散方程式系を定義し,反応拡散方程式系のモデルパラメータとパーシステントホモロジーの間の関係性を議論する。Section \ref{sec:example}では,Section \ref{sec:method}での議論を代表的な反応拡散方程式系であるGray--Scottモデルに適用する。Section \ref{sec:simulation}では,Section \ref{sec:example}で得られた結果を数値実験を利用して視覚化する。最後に,Section \ref{sec:conclusion}では,本論文で得られた結果とその限界,および,今後の研究課題を議論する。

% 第 3 節では代数的トポロジーに基づく計算手法の基礎を導入し、後に CIMA 反応系と Schnakenberg 反応系の解のデータに適用する。これらのトポロジカルな要約(バーコード)を Turing 空間の複数の点で計算した後、第 4 節では得られたパターンをクラスタリングするアルゴリズムを説明し、その結果のパラメータ推定やモデル選択への応用を述べる。第 5 節では、ここで扱う例に関してはバーコードがパターンを分類するのに十分であることを示し、第 6 節で結果の限界と今後の研究課題を議論する。

\section{Preliminary}\label{sec:preliminary}

\subsection{Persistent Homology}

\paragraph{ホモロジー}
ホモロジーに関する数学的に厳密な定義は与えないが,ホモロジーが位相空間の「穴」や「空洞」を計算できることを直観的に説明する。また,円周$S^1$やトーラス$T^2$などの代表的な位相空間のホモロジー群を紹介する。このようにして,最低限のホモロジーの概念を導入する。

\paragraph{パーシステントホモロジー}

パーシステントホモロジーがホモロジーを利用してデータの「形状」を捉えるデータ解析技術であることを述べる。その後パーシステントホモロジーの定義を述べる。まず,位相空間上$X$の関数$f: X \to \mathbb{R}$に対して,劣位集合フィルトレーションを定義して,劣位集合フィルトレーションの各点のホモロジーとその間の線型写像の組の系列としてパーシステントホモロジーを説明する。

\paragraph{パーシステンス加群}

パーシステントホモロジーの一般化としてパーシステンス加群を導入する。パーシステントホモロジーが実際にパーシステンス化群であることを述べる。

\paragraph{パーシステンス図}

パーシステンス加群の分解定理を通してパーシステンス図を定義する。また,パーシステントホモロジーにおけるパーシステンス図の直感的な意味を述べる。

\paragraph{q-tame性}

パーシステンス加群のq-tame性を定義する。また,三角形分割可能な位相空間上の連続関数から得られるパーシステントホモロジーはq-tameであることを述べる。

\paragraph{q-tame性による安定性定理}

パーシステンス図の間の距離としてボトルネック距離を定義する。さらに,q-tameなパーシステントホモロジーに対して成り立つボトルネック距離の安定性を述べる。

\subsection{Turing pattern and reaction--diffusion systems}

\paragraph{Turingパターンと反応拡散系}

動植物が持つ特徴的な体表パターンの一部が反応拡散系によって説明できることを具体例とともに紹介する。反応拡散系は連立偏微分方程式
\begin{align*}
  \dfrac{ \partial u_i }{ \partial t } \left( x, t \right) = D_i \Delta u_i \left( x, t \right) + R_i \left( u_1, \cdots, u_n; \theta \right) \quad \left( 1 \leq i \leq n \right)
\end{align*}
で与えられることも説明しておく。

\subsection{Semigroup method}

偏微分方程式の抽象的解法である半群法に関する必要最低限の内容を整理する。

\section{Method}\label{sec:method}

\subsection{Problem setting}

本論文で検討する反応拡散方程式系を定義する。具体的には,反応拡散方程式系
\begin{align*}
  \dfrac{ \partial u_i }{ \partial t } \left( x, t \right) = D_i \Delta u_i \left( x, t \right) + R_i \left( u_1, \cdots, u_n; \theta \right) \quad \left( 1 \leq i \leq n \right)
\end{align*}
を周期境界条件を付与した$\left[ 0, 1 \right]^d$上で考えることを述べる。さらに,本論文の主定理を明示しておく。具体的には,ある条件のもとで,異なる二つのモデルパラメータ$p = \left( D_1, \cdots, D_n, \theta \right)$および$p^\prime = \left( D_1^\prime, \cdots, D_n^\prime, \theta^\prime \right)$に対するモデルの解を$\left( u_1, \cdots, u_n \right)$および$\left( u_1^\prime, \cdots, u_n^\prime \right)$とするとき,
\begin{align*}
  \sup_{0 \leq t \leq T} d_B \left( \mathrm{dgm}~u_i, \mathrm{dgm}~u_i^\prime \right) \leq C \left\| p - p^\prime \right\|_\infty
\end{align*}
が成り立つことを示す。ただし,初期値は共通であるとする。

\subsection{Bounds for solution differences under parameter perturbations}

モデル解のLipschitz性
\begin{align*}
  \sup_{0 \leq t \leq T} \left\| u_i - u_i^\prime \right\|_\infty \leq C\left\| p - p^\prime \right\|_\infty
\end{align*}
を示す。

\subsection{Bounds for persistent diagram differences under parameter perturbations}

モデル解に対してパーシステントホモロジーの安定性定理
\begin{align*}
  \sup_{0 \leq t \leq T} d_B \left( \mathrm{dgm}~u_i, \mathrm{dgm}~u_i^\prime \right) \leq \left\| u_i - u_i^\prime \right\|_\infty
\end{align*}
が成り立つことを示す。したがって,主定理が成り立つことを示す。

\section{Example: Gray--Scott Model}\label{sec:example}

Section \ref{sec:method}で示した内容が,代表的な拡散反応系であるGray--Scottモデルにも適用できることを示す。

\section{Simulation Studies}\label{sec:simulation}

\subsection{Visualizations of Persistence Diagrams}

いろいろなパラメータに対してGray--Scottモデルのパターンおよびそこから得られるパーシステンス図を視覚化して,類似したモデルパラメータからは類似したパーシステンス図が得られることを視覚的に紹介する。時間発展も見てみる。

\subsection{Numerical Evaluation of Parameter Lipschitz Constant}

Gray--Scottモデルについて,Section \ref{sec:method}で示したLipschitz性が成り立つことを数値実験で示す。また,数値計算で求まるLipschitz定数とSection \ref{sec:method}で導出したLipschitz定数の比較も行う。

\section{Conclusion}\label{sec:conclusion}

本論文の主定理
\begin{align*}
  \sup_{0 \leq t \leq T} d_B \left( \mathrm{dgm}~u_i, \mathrm{dgm}~u_i^\prime \right) \leq C\left\| p - p^\prime \right\|_\infty
\end{align*}
によって,パラメータの微小な摂動に対して,パーシステントホモロジーが安定的であることが理解できることを述べ,研究目標が達成されたことを述べる。一方で,Lipschitz定数$C$が時間発展的に大きくなることを述べて,実用的な限界について述べる。最後にあれば今後の展望を述べる。

\printbibliography

\end{document}