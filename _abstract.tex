\clearpage
\thispagestyle{empty}
\null\vfill
\begin{center}
  {\Large \headfont Abstract}
\end{center}

% Turingパターンは生物の体表パターン形成を説明する重要な数理モデルである。近年,Turingパターンを定量的に比較・解析する手法として,パーシステントホモロジーに基づくトポロジカルデータ解析が用いられている。本論文では,Turingパターンを生成する拡散反応系について,ある条件のもとで,モデルパラメータの摂動に対してパーシステンス図がリプシッツ連続に依存することを証明する。さらに,代表的な拡散反応系であるGray--Scottモデルが同条件を満たしうることを示し,本結果の適用可能性を具体例により裏付ける。数値シミュレーションでは,パラメータを変化させた際のボトルネック距離の挙動が理論結果と整合的が示される。

Turing patterns are an important mathematical model for biological body surface pattern formation. Recently, persistent homology has been used to quantitatively compare and analyze such patterns. In this paper, we prove that, under certain conditions, the persistence diagrams obtained from Turing patterns are Lipschitz continuous with respect to the model parameters of the reaction-diffusion system that generates them. We further show that the Gray--Scott model, a representative reaction-diffusion system, can satisfy these conditions, thereby illustrating the applicability of our theory. In numerical simulations for the Gray--Scott model, we demonstrate that the theoretical results indeed hold in practice, and we also visually confirm the correspondence between similarity in the parameters and similarity in the persistence diagrams.

\vfill\vfill\vfill
\cleardoublepage