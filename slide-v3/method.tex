\begin{frame}
  \frametitle{Turingパターン}

  自然界には特徴的な体表パターンを持つ動物が存在する。

  例えば,キリンは網目状の模様,ジャガーは斑点模様,シマウマは縞模様を特徴としている。
  \vspace{0.8 cm}
  \begin{figure}[h]
    \begin{minipage}{0.32\linewidth}
      \includegraphics[width=\linewidth]{figures/giraffe.jpg}
    \end{minipage}
    \hfill
    \begin{minipage}{0.32\linewidth}
      \includegraphics[width=\linewidth]{figures/jaguar.jpg}
    \end{minipage}
    \hfill
    \begin{minipage}{0.32\linewidth}
      \includegraphics[width=\linewidth]{figures/zebra.jpg}
    \end{minipage}
  \end{figure}

\end{frame}

\begin{frame}
  \frametitle{Turingパターン}

  A. M. Turingは,このようなパターンがモルフォゲンと総称される化学物質の相互作用(拡散・反応)によって生じると提唱した\cite{turing1990chemical}。

  すなわち,体表パターンは$n \geq 2$種類のモルフォゲンが以下のような拡散反応方程式に従うことによって生じる。
  \begin{align*}
    \dfrac{ \partial u_i }{ \partial t } = D_i \Delta u_i + f_i \left( u_1, \cdots, u_n \right) ~ \left( 1 \leq i \leq n \right)
  \end{align*}

  ここで,$u_i$は各モルフォゲンの濃度,$f_i$は反応項,$\Delta u_i$は拡散項を表している。

  拡散反応方程式系によって生じるパターンは一般に\alert{Turingパターン}と呼ばれる。

\end{frame}

\begin{frame}
  \frametitle{Turingパターン}

  海水エンゼルフィッシュ,シマウマ,ヒョウ,ジャガーといったさまざまな生物のパターンが拡散反応方程式系で再現できることが示されている\cite{kondo1995reaction, gravan2004evolving, liu2006two}。

  \begin{figure}[h]
    \centering
    \includegraphics[width=15 cm]{figures/angelfish.png}
    \caption{海水エンゼルフィッシュの体表パターンと対応するTuringパターン\cite{kondo1995reaction}}
  \end{figure}

\end{frame}

\begin{frame}
  \frametitle{Turingパターン}

  Turingパターンは、生物の体表に見られる多様な模様がどのようにして形成されるかを説明するための理論的枠組みであり、形態形成を理解するという意味で生命科学において極めて重要なモデルである。

  さらに、特徴的な空間パターンを伴う皮膚病の解明という観点から医学的にも一部で関心が高く、多孔質構造の自己組織化などを実現する点では、材料科学においても重要な役割を果たしている。
  
\end{frame}

\begin{frame}
  \frametitle{パーシステントホモロジー}

  \cite{volkening2024methods}では,拡散反応方程式系が生成するパターンデータ対する定量的解析手法として\alert{トポロジカルデータ解析}が紹介されている。

  トポロジカルデータ解析は,データの「形状」を定量的に調べる手法として近年大きく発展している\cite{carlsson2009topology}。
  
  \alert{パーシステントホモロジー}はトポロジカルデータ解析の代表的手法である\cite{edelsbrunner2002topological}。

\end{frame}

\begin{frame}
  \frametitle{パーシステントホモロジー}

  パーシステントホモロジーでは,データを時系列的に変化させることによって,データの位相的特徴(ホモロジー)の時系列的な変化を取得する。

  \begin{center}
  \begin{tikzpicture}[scale=0.8]
    \def\R{2.0}
    \def\r{0.6}
    \def\nptsB{100}
    \def\nptsT{40}
    \def\dx{6.8}
    \def\pointsize{0.05}
    \def\labelsep{1.8}

    \pgfmathsetmacro{\rlabel}{- \R - \labelsep}
    \pgfmathsetmacro{\llabel}{- \R - \labelsep - 1}

    \pgfmathsetseed{20251010}

    \foreach \i in {1,...,\nptsB}{
      \pgfmathsetmacro{\ang}{360*rnd}
      \global\expandafter\edef\csname angleb\i\endcsname{\ang}
    }

    \foreach \i in {1,...,\nptsT}{
      \pgfmathsetmacro{\ang}{360*rnd}
      \global\expandafter\edef\csname anglet\i\endcsname{\ang}
    }

    \newcommand{\drawEight}[4]{%
      \coordinate (Cb) at (#1, 0);
      \coordinate (Ct) at (#1, {\R + \r + 0.5});

      \foreach \i in {1,...,\nptsB}{
        \path (Cb) ++(\csname angleb\i\endcsname:\R) coordinate (P);
        \fill[accent] (P) circle[radius=#2];
      }

      \foreach \i in {1,...,\nptsT}{
        \path (Ct) ++(\csname anglet\i\endcsname:\r) coordinate (Q);
        \fill[accent] (Q) circle[radius=#2];
      }

      \node at (#1, \rlabel) {#3};
      \node at (#1, \llabel) {#4};
    }


    \drawEight{0}{\pointsize}{$a = 1$}{ループ: 0}
    \drawEight{\dx}{4.4 * \pointsize}{$a = 5$}{ループ: 1}
    \drawEight{2 * \dx}{7.8 * \pointsize}{$a = 8$}{ループ: 2}
    \drawEight{3 * \dx}{12 * \pointsize}{$a = 12$}{ループ: 1}

  \end{tikzpicture}
\end{center}
\end{frame}

\begin{frame}
  \frametitle{パーシステントホモロジー}

  パターン形成系におけるパーシステントホモロジーの応用に関する研究はすでに多く行われている\cite{topaz2015topological, nardini2021topological, stolz2022multiscale, mcdonald2023zigzag, mcguirl2020topological, yang2025topological, kanamori2025analysis, spector2024persistent}。

  たとえば,\cite{kanamori2025analysis}では実際の魚類の体表パターンをパーシステントホモロジーで分類しており,\cite{spector2024persistent}ではパーシステントホモロジーを活用したパターンのクラスタリングを行っている。

\end{frame}

\begin{frame}
  \frametitle{パーシステントホモロジー}

  これらの研究は,パーシステントホモロジーがパターン解析に有用であることを示している。したがって,パターン生成系とパーシステントホモロジーの間には何かしら数学的構造が存在していると期待される。

  本研究では,具体的な拡散反応方程式系であるGray--Scottモデルをパーシステントホモロジーで理論的に解析する。

  特に,\cite{kanamori2025analysis}と\cite{spector2024persistent}の研究結果に基づいて,モデルパラメータがパーシステントホモロジーに与える影響を主に解析する。

\end{frame}

\begin{frame}
  \frametitle{Gray--Scottモデル}

  P. GrayとS.K. Scottは,\cite{gray1983autocatalytic}および\cite{gray1984autocatalytic}において,以下のような自己触媒過程のモデル解析を行った。
  \begin{align*}
    U + 2V \to 3V,~V \to P
  \end{align*}
  この反応式は拡散反応方程式として以下のようにモデル化される。
  \begin{align*}
    \partial_t u = D_1 \Delta u - u v^2 + F \left( 1 - u \right) \\
    \partial_t v = D_2 \Delta v + u v^2 - \left( F + k \right) v
  \end{align*}
  このような拡散反応方程式は一般に\alert{Gray--Scottモデル}と呼ばれる。
  
  ただし,$u, v$はそれぞれ$U, V$の濃度を表しており,$\left( D_1, D_2, F, k \right)$はモデルパラメータである。
  
\end{frame}

\begin{frame}
  \frametitle{Gray--Scottモデル}

  Gray--Scottモデルは,Turingパターンを生成する代表的な拡散反応方程式である。モデルパラメータ$\left( D_1, D_2, F, k \right)$に応じていくつかのパターンを生成することが知られている。

    \begin{figure}[h]
    \centering
    \includegraphics[width=11 cm]{figures/gray-scott-pattern.png}
    \caption{Gray--Scottモデルが生成するTuringパターンの具体例\cite{上山大信2010gray}}
  \end{figure}

\end{frame}

\begin{frame}
  \frametitle{パーシステントホモロジー}

  パーシステントホモロジーのおいて,データの時系列的変化は集合の増大列として表現される。

  \begin{dfn*}
    $X$を位相空間とし,$f: X \to \mathbb{R}$を$X$上の実数値関数とするとき,任意の$r \in \mathbb{R}$に対して,
    \begin{align*}
      X \left( f \right)_r = \Set{ x \in X | f\left( x \right) \leq r }
    \end{align*}
    とおく。このとき,$X$の部分位相空間の増大列
    \begin{align*}
      \mathcal{F} \left[ X \left( f \right) \right] = \left( X \left( f \right)_r \right)_{r \in \mathbb{R}}
    \end{align*}
    を$X$の$f$に関する劣位集合フィルトレーションという。
  \end{dfn*}

\end{frame}

\begin{frame}
  \frametitle{パーシステントホモロジー}

  パーシステントホモロジーでは,増大列(劣位集合フィルトレーション)の各点に対してホモロジー群$H_*\left( \cdot \right)$とホモロジー群の間の準同型$\left(  i_{\cdot, \cdot}\right)_*$を組み合わせることにより,ホモロジーの変化を追跡する。

  \begin{dfn*}
    $X$を位相空間とし,$f: X \to \mathbb{R}$を$X$上の実数値関数とする。また,任意の$r \leq s$に対して,包含写像$i_{r, s}: X \left( f \right)_r \to X \left( f \right)_s$が誘導する準同型を$\left( i_{r, s} \right)_*: H_* \left( X \left( f \right)_s \right) \to H_* \left( X \left( f \right)_s \right)$とかく。このとき,任意の負でない整数$n \geq 0$に対して,
    \begin{align*}
      H_n \left( \mathcal{F} \left[ X \left( f \right) \right] \right) = \left( \left( H_n \left( X \left( f \right)_r \right) \right)_{r\in\mathbb{R}}, \left( \left( i_{r, s} \right)_* \right)_{r \leq s} \right)
    \end{align*}
    を$X$の$f$に関する劣位集合フィルトレーション$\mathcal{F} \left[ X \left( f \right) \right]$の$n$次元パーシステントホモロジーという。
  \end{dfn*}

\end{frame}

\begin{frame}
  \frametitle{パーシステントホモロジー}

  \begin{eg*}
    距離空間$\left( X, d \right)$上の離散点集合$P$と点$x$に対して,$f_P \left( x \right) = d \left( P, x \right)$とおく。このとき,$X \left( f_P \right)_r = \bigcup_{p \in P} \Set{ d \left( p, x \right) \leq r }$となるから,$f_P$に関するパーシステントホモロジーは,以下の図形のホモロジー(連結成分,穴,空洞)を調べている。
  \end{eg*}

  \begin{center}
  \begin{tikzpicture}[scale=0.8]
    \def\R{2.0}
    \def\r{0.6}
    \def\nptsB{100}
    \def\nptsT{40}
    \def\dx{6.8}
    \def\pointsize{0.05}
    \def\labelsep{1.8}

    \pgfmathsetmacro{\rlabel}{- \R - \labelsep}
    \pgfmathsetmacro{\llabel}{- \R - \labelsep - 1}

    \pgfmathsetseed{20251010}

    \foreach \i in {1,...,\nptsB}{
      \pgfmathsetmacro{\ang}{360*rnd}
      \global\expandafter\edef\csname angleb\i\endcsname{\ang}
    }

    \foreach \i in {1,...,\nptsT}{
      \pgfmathsetmacro{\ang}{360*rnd}
      \global\expandafter\edef\csname anglet\i\endcsname{\ang}
    }

    \newcommand{\drawEight}[4]{%
      \coordinate (Cb) at (#1, 0);
      \coordinate (Ct) at (#1, {\R + \r + 0.5});

      \foreach \i in {1,...,\nptsB}{
        \path (Cb) ++(\csname angleb\i\endcsname:\R) coordinate (P);
        \fill[accent] (P) circle[radius=#2];
      }

      \foreach \i in {1,...,\nptsT}{
        \path (Ct) ++(\csname anglet\i\endcsname:\r) coordinate (Q);
        \fill[accent] (Q) circle[radius=#2];
      }

      \node at (#1, \rlabel) {#3};
      \node at (#1, \llabel) {#4};
    }


    \drawEight{0}{\pointsize}{$a = 1$}{ループ: 0}
    \drawEight{\dx}{4.4 * \pointsize}{$a = 5$}{ループ: 1}
    \drawEight{2 * \dx}{7.8 * \pointsize}{$a = 8$}{ループ: 2}
    \drawEight{3 * \dx}{12 * \pointsize}{$a = 12$}{ループ: 1}

  \end{tikzpicture}
\end{center}
\end{frame}

\begin{frame}
  \frametitle{パーシステンス図}

  \begin{minipage}{0.53\linewidth}
    パーシステントホモロジーは数学的に扱いにくいので,\alert{パーシステンス図}という集合を用いて要約する。
    
    ホモロジーとして「穴」を調べる場合,「穴」が生成した時刻を$x$軸に,消滅した時刻を$y$軸にプロットして作成する。

    パーシステンス図の間には\alert{ボトルネック距離$d_B$}という距離が存在する。
  \end{minipage}
  \hfill
  \begin{minipage}{0.44\linewidth}
    \begin{center}
  \begin{tikzpicture}
    \draw[->] (-1,0) -- (8,0) node[below right] {birth};
    \draw[->] (0,-1) -- (0,8) node[above left] {death};
    \draw[gray, dashed, line width=0.8pt] (0,0) -- (7.5,7.5) node[above right, black] {$y = x$};

    \newcommand{\pdpoint}[2]{%
      \fill[accent] (#1,#2) circle (1 mm);
    }

    \pdpoint{1.2}{3.8}
    \pdpoint{2.0}{5.0}
    \pdpoint{4.5}{7.0}
    \pdpoint{6.0}{7.2}

    \draw[gray, dashed] (4.5,7.0) -- (4.5,0);
    \draw[gray, dashed] (4.5,7.0) -- (0,7.0);

    \fill[accent] (4.5,0) circle (0.1mm) node[below, black] {$b$};
    \fill[accent] (0,7.0) circle (0.1mm) node[left, black] {$d$};
  \end{tikzpicture}
\end{center}
  \end{minipage}
\end{frame}

\begin{frame}
  \frametitle{パーシステントホモロジーの安定性定理}
  $X$上の関数$f:X \to \mathbb{R}$に対して,$f$から定まるパーシステンス図を$\mathrm{dgm}~f$と書くことにする。

  \begin{thm*}[\cite{cohen2005stability}]
    $X$を位相空間とする。二つの従順な関数$f, g: X \to \mathbb{R}$に対して,
    \begin{align*}
      d_B \left( \mathrm{dgm}~f, \mathrm{dgm}~g \right) \leq \left\| f - g \right\|_{\infty}
    \end{align*}
    が成り立つ。
  \end{thm*}

\end{frame}

\begin{frame}
  \frametitle{Contribution}

  $K$を$\left( 0, \infty \right)^4$上のコンパクト集合とする。$\theta = \left( D_1, D_2, F, k \right) \in K$をパラメータとするGray-Scottモデルを
  \begin{align*}
    \partial_t u_\theta = D_1 \Delta u_\theta - u_\theta v_\theta^2 + F \left( 1 - u \right) \\
    \partial_t v_\theta = D_2 \Delta v_\theta + u_\theta v_\theta^2 - \left( F + k \right) v
  \end{align*}
  と表す。

  以下では,境界のない正方形(2次元トーラス)上でGray-Scottモデルを考える。

\end{frame}

\begin{frame}
  \frametitle{Contribution}

  \begin{thm*}
    $T > 0$とする。ある$M > 0$が存在して,任意の$\theta = \left( D_1, D_2, F, k \right) \in K$および$\theta^\prime =  \left( D_1^\prime, D_2^\prime, F^\prime, k^\prime \right) \in K$に対して,
    \begin{align*}
        \sup_{0 \leq t \leq T} \left\| u_{\theta} \right\|_\infty \leq M,~ \sup_{0 \leq t \leq T} \left\| u_{\theta^\prime} \right\|_\infty \leq M
    \end{align*}
    が成り立つと仮定する。このとき,$\theta, \theta^\prime$に依存しない定数$C$が存在して
    \begin{align*}
      \sup_{0 \leq t \leq T} d_B \left( \mathrm{dgm}~u_\theta, \mathrm{dgm}~u_{\theta^\prime} \right) \leq C \left\| \theta - \theta^\prime \right\|_\infty
    \end{align*}
    が成り立つ。
  \end{thm*}

\end{frame}

\begin{frame}
  \frametitle{Method}

  主定理の証明は,半群法に基づいて解を表示することによって行う。

  \begin{thm*}[\cite{pazy2012semigroups}]
    $\tilde{U} = \left( \tilde{u}, \tilde{v} \right)$を初期値とするGray--Scottモデル
    \begin{align*}
      \partial_t u_\theta = D_1 \Delta u_\theta - u_\theta v_\theta^2 + F \left( 1 - u \right) \\
      \partial_t v_\theta = D_2 \Delta v_\theta + u_\theta v_\theta^2 - \left( F + k \right) v
    \end{align*}
    の解$\left( u_\theta, v_\theta \right)$は
    \begin{align*}
      u_\theta \left( \cdot, t \right) &= e^{D_1 \Delta t} \tilde{u} + \int_0^t e^{D_1 \Delta t} \left( t - s \right) \left( - u_\theta v_\theta^2 + F \left( 1 - u_\theta \right) \right) \left( s \right) ds \\
      v_\theta \left( \cdot, t \right) &= e^{D_1 \Delta t} \tilde{v} + \int_0^t e^{D_1 \Delta t} \left( t - s \right) \left( u_\theta v_\theta^2 - \left( F + k \right) v_\theta \right) \left( s \right) ds
    \end{align*}
    の形で与えられる。
  \end{thm*}

\end{frame}

\begin{frame}
  \frametitle{Method}

  解の表示
  \begin{align*}
    u_\theta \left( \cdot, t \right) &= e^{D_1 \Delta t} \tilde{u} + \int_0^t e^{D_1 \Delta t} \left( t - s \right) \left( - u_\theta v_\theta^2 + F \left( 1 - u_\theta \right) \right) \left( s \right) ds
  \end{align*}
  を利用すると,
  \begin{align*}
    \left\| u_\theta - u_{\theta^\prime} \right\|_\infty \leq C \left\| \theta - \theta^\prime \right\|_\infty
  \end{align*}
  という不等式が構成できる。これを安定性定理と組み合わせることにより主定理を得る。

\end{frame}

\begin{frame}
  \frametitle{Discussion}

  本研究の主定理
  \begin{align*}
    \sup_{0 \leq t \leq T} d_B \left( \mathrm{dgm}~u_\theta, \mathrm{dgm}~u_{\theta^\prime} \right) \leq C \left\| \theta - \theta^\prime \right\|_\infty
  \end{align*}
  によって,十分に近い二つのモデルパラメータを持つGray--Scottモデルは,パーシステントホモロジーの意味で十分に近いパターンを生成するということが示される。

  この結論は,\cite{kanamori2025analysis}および\cite{spector2024persistent}の数値実験的な結果をサポートするものである。

\end{frame}

\begin{frame}
  \frametitle{Discussion}

  主定理の主張においては,
  \begin{align*}
    \sup_{0 \leq t \leq T} \left\| u_{\theta} \right\|_\infty \leq M,~ \sup_{0 \leq t \leq T} \left\| u_{\theta^\prime} \right\|_\infty \leq M
  \end{align*}
  という仮定を置いた。この$M$は実際にはどのような形で存在するかが不明瞭であり,仮定としては妥当とは言えない。

  一方で,この$M$は初期値$\tilde{u}$を利用して表現できると考えており,現在はその証明を行っている。

\end{frame}
